\documentclass[12pt]{article}
\textwidth=17cm \oddsidemargin=-0.9cm \evensidemargin=-0.9cm
\textheight=23.7cm \topmargin=-1.7cm

\usepackage{amssymb, amsmath, amsfonts}
\usepackage{moreverb}
\usepackage{graphicx}
\usepackage{enumerate}
\usepackage{graphics}
\usepackage{color}
\usepackage{array}
\usepackage{float}
\usepackage{hyperref}
\usepackage{textcomp}
\usepackage{alltt}
\usepackage{mathtools}
\usepackage{tikz}
\usepackage{bigints}
\newcommand{\suchthat}{\, \mid \,}
\renewcommand{\theenumi}{\alph{enumi}}
\setcounter{section}{-1}
%\setlength{\jot}{30pt}


\begin{document}

\begin{center}
	{\bf\LARGE The Ecological Effects of Trait Variation in a\\ \vskip 5pt$u$-Predator, $v$-Prey System \small (draft)}\\ \vskip 3pt \rule{4cm}{0.4pt}
	\vskip 5pt
	Sam Fleischer, Pablo Chavarria, Casey terHorst, Jing Li \\ Start Date:  March 2014 - - Today's Date: \today \rm
\end{center}

\vskip 15pt

\section*{The Model}
\subsection*{Attack Rate as a Function of Predator and Prey Trait Values}
Let $M_i(t)$ be the density of the $i$\textsuperscript{th} predator species, and let $N_j(t)$ be the density of the $j$\textsuperscript{th} prey species.  Let $\overline{m_i}(t)$ be the mean of a single quantitative trait in the $i$\textsuperscript{th} predator species, and let $\overline{n_j}(t)$ be the mean of a single quantitative trait in the $j$\textsuperscript{th} prey species.  Suppose the traits are normally distributed, and stay normally distributed, with $\sigma_i^2$ as the constant variance of the $i$\textsuperscript{th} predator species, and with $\beta_j^2$ as the constant variance of the $j$\textsuperscript{th} prey species.  Their distributions are given by:
\begin{align*}
	p(m_i, \overline{m_i}) &= \frac{1}{\sqrt{2\pi\sigma_i^2}}\exp\left[{-\frac{(m_i - \overline{m_i})^2}{2\sigma_i^2}}\right] \\
	p(n_j, \overline{n_j}) &= \frac{1}{\sqrt{2\pi\beta_j^2}}\exp\left[{-\frac{(n_j - \overline{n_j})^2}{2\beta_j^2}}\right]
\end{align*}

\noindent Assume all of the species' phenotypic variances have genetic and environment components:
\begin{align*}
	\sigma_i^2 = \sigma_{Gi}^2 + \sigma_{Ei}^2 \\
	\beta_j^2 = \beta_{Gj}^2 + \beta_{Ej}^2
\end{align*}

\noindent Let $a_{ij}(m_i, n_j)$ be the attack rate of an individual predator from species $i$ on an individual prey from species $j$.  Supposing the attack rate is optimal at $\alpha_{ij}$ when the predator's trait and prey's trait are at an optimal difference $\theta_{ij}$, and decreases in a Gaussian manner as the trait's deviate from that difference, then
\begin{align*}
	a_{ij}(m_i, n_j) = \alpha_{ij} \exp\left[{-\frac{((m_i - n_j) - \theta_{ij})^2}{2\tau_{ij}^2}}\right]
\end{align*}

\noindent where $\tau_{ij}$ determines how phenotypically specialized a predator individual of species $i$ must be to use a prey individual of species $j$.  Let $\overline{a_{ij}}(\overline{m_i}, \overline{n_j})$ be the mean attack rate of predator species $i$ on prey species $j$.  Thus,
\begin{align*}
	\overline{a_{ij}}(\overline{m_i}, \overline{n_j}) &= \int_{-\infty}^{\infty}\int_{-\infty}^{\infty} a_{ij}(m_i, n_j) \cdot p(m_i, \overline{m_i}) \cdot p(n_j, \overline{n_j}) dm_i dn_j \\
	&= \frac{\alpha_{ij}\tau_{ij}}{\sqrt{\sigma_i^2 + \beta_j^2 + \tau_{ij}^2}}\exp\left[{-\frac{((\overline{m_i} - \overline{n_j}) - \theta_{ij})^2}{2(\sigma_i^2 + \beta_j^2 + \tau_{ij}^2)}}\right]
\end{align*}

\subsection*{Fitness Assumptions}
\noindent Let $u$ be the number of predator species, and let $v$ be the number of prey species.  Assuming predators have a linear functional response, convert the consumed prey into offspring with efficiencies $e_{ij}$, and experience a per-capita mortality rate $d_i$, then the fitness of a predator with phenotype $m_i$ is
\begin{align*}
	W_i(m_i, [N]_1^v, [n]_1^v) &= \sum_{j = 1}^v\left(e_{ij}a_{ij}(m_i, n_j)N_j\right) - d_i
\end{align*}

\noindent and thus the mean fitness of the $i$\textsuperscript{th} predator population is
\begin{align*}
	\overline{W_i}(\overline{m_i}, [N]_1^v, [\overline{n}]_1^v) &= \int\limits_{\mathbb{R}^{v+1}} W_i(m_i, [N]_1^v, [n]_1^v) p(m_i, \overline{m_i}) \prod\limits_{j = 1}^{v}\left[p(n_j, \overline{n_j})\right] dm_i \prod\limits_{j = 1}^{v}\left[dn_j\right] \\
	&= \sum_{j=1}^v\left(e_{ij}\overline{a_{ij}}(\overline{m_i}, \overline{n_j})N_j\right) - d_i
\end{align*}

\noindent Suppose prey species $j$ experiences logistic-type growth in the absence of predators with carrying capacity $K_j$ and intrinsic growth rate $r_j$.  However, assume the intrinsic growth rate varies as a function of the prey individual's trait value.  Assume the contribution of a prey individual to its population decreases in a Gaussian manner as the trait value deviates away from an optimal trait value for that species, $\phi_j$.  Let $\rho_j$ be the maximal contribution rate and $\gamma_j$ be the ``cost variance".  In other words,

\begin{align*}
	r_j(n_j) = \rho_j\exp{\left[-\frac{(n_j - \phi_j)^2}{2\gamma_j^2}\right]}
\end{align*}

\noindent Thus, the average intrinsic growth rate for the prey population is

\begin{align*}
	\overline{r}_j(\overline{n}_j) &= \int_{-\infty}^\infty{r_j(n_j)p(n_j, \overline{n}_j)}dn_j \\
	&= \frac{\rho_j\gamma_j}{\sqrt{\beta_j^2 + \gamma_j^2}}\exp\left[{-\frac{(n_j - \phi_j)^2}{2(\beta_j^2 + \gamma_j^2)}}\right]
\end{align*}

\noindent Define the fitness of prey individuals with phenotype $n_j$ as
\begin{align*}
	Y_j(N_j, n_j, [M]_1^u, [m]_1^u) &= r_j(n_j)\left(1 - \frac{N_j}{K_j} \right) - \sum_{i = 1}^u\left(a_{ij}(m_i, n_j)M_i\right) \\
	&= \rho_j\exp{\left[-\frac{(n_j - \phi_j)^2}{2\gamma_j^2}\right]}\left(1 - \frac{N_j}{K_j}\right) - \sum_{i = 1}^u\left(a_{ij}(m_i, n_j)M_i\right)
\end{align*}

\noindent and thus the mean fitness of the $j$\textsuperscript{th} prey population is
\begin{align*}
	\overline{Y_j}(N_j, \overline{n_j}, [M]_1^u, [\overline{m}]_1^u) &= \int\limits_{\mathbb{R}^{u+1}} Y_j(N_j, n_j, [M]_1^u, [m]_1^u) \prod\limits_{i=1}^{u}\left[p(m_i, \overline{m_i})\right] p(n_j, \overline{n_j}) \prod\limits_{i=1}^{u}\left[dm_i\right] dn_j \\
	&= \overline{r}_j(\overline{n}_j) \left(1 - \frac{N_j}{K_j} \right) - \sum_{i = 1}^u \overline{a_{ij}}(\overline{m_i}, \overline{n_j}) M_i
\end{align*}

\subsection*{Ecological Dynamics}
\noindent The ecological dynamics of the model (population densities) are given by
\begin{align}
	\begin{cases}
		\dfrac{dM_i}{dt} &= M_i \overline{W_i}(\overline{m_i}, [N]_1^v, [\overline{n}]_1^v) \\[.25cm]
		\dfrac{dN_j}{dt} &= N_j \overline{Y_j}(N_j, \overline{n_j}, [M]_1^u, [\overline{m}]_1^u)
	\end{cases}
\end{align}

\subsection*{Evolutionary Dynamics}
\noindent Assume the evolution of the mean trait value is always in the direction which increases the mean fitness in the population.  Thus the evolutionary dynamics are given by
\begin{align}
	\begin{cases}
		\dfrac{d\overline{m_i}}{dt} &= \sigma_{Gi}^2 \; \dfrac{\partial \overline{W_i}}{\partial \overline{m_i}} \\[.25cm]
		\dfrac{d\overline{n_j}}{dt} &= \beta_{Gj}^2 \; \dfrac{\partial \overline{Y_j}}{\partial \overline{n_j}}
	\end{cases}
\end{align}

\noindent where

\begin{align*}
	\frac{\partial \overline{W_i}}{\partial \overline{m_i}} &= \sum_{j=1}^v\left[\frac{e_{ij}N_j(\theta_{ij} - (\overline{m}_i - \overline{n}_j))}{\sigma_i^2 + \beta_j^2 + \tau_{ij}^2} \cdot \overline{a}_{ij}(\overline{m}_i, \overline{n}_j)\right]\\
	\frac{\partial \overline{Y_j}}{\partial \overline{n_j}} &= \overline{r}_j(\overline{n}_j)\left(1 - \frac{N_j}{K_j}\right)\frac{(\phi_j - \overline{n}_j)}{\beta_j^2 + \gamma_j^2} + \sum_{i=1}^u\left[\frac{M_i(\theta_{ij} - (\overline{m}_i - \overline{n}_j))}{\sigma_i^2 + \beta_j^2 + \tau_{ij}^2} \cdot \overline{a}_{ij}(\overline{m}_i, \overline{n}_j)\right]
\end{align*}

\subsection*{The $1\times1$ Model}
\noindent For example, the $1\times1$ model is a four-dimensional system given by
\begin{align*}
	\left\{\begin{array}{lll}
		f_1 = \dfrac{dM}{dt} &= M \overline{W}(\overline{m}, N, \overline{n}) &= M \left[e\overline{a}(\overline{m}, \overline{n})N - d\right] \\[.5cm]
		f_2 = \dfrac{dN}{dt} &= N \overline{Y}(N, \overline{n}, M, \overline{m}) &= N \left[\overline{r}(\overline{n}) \left(1 - \dfrac{N}{K} \right) - \overline{a}(\overline{m}, \overline{n}) M\right] \\[.5cm]
		f_3 = \dfrac{d\overline{m}}{dt} &= \sigma_{G}^2 \; \dfrac{\partial \overline{W}}{\partial \overline{m}} &= \sigma_{G}^2 \left[\dfrac{eN(\theta - (\overline{m} - \overline{n}))}{\sigma^2 + \beta^2 + \tau^2} \cdot \overline{a}(\overline{m}, \overline{n})\right]\\[.5cm]
		f_4 = \dfrac{d\overline{n}}{dt} &= \beta_{G}^2 \; \dfrac{\partial \overline{Y}}{\partial \overline{n}} &= \beta_{G}^2 \left[\overline{r}(\overline{n})\left(1 - \dfrac{N}{K}\right)\dfrac{(\phi - \overline{n})}{\beta^2 + \gamma^2} + \dfrac{M(\theta - (\overline{m} - \overline{n}))}{\sigma^2 + \beta^2 + \tau^2} \cdot \overline{a}(\overline{m}, \overline{n})\right]
	\end{array}\right.
\end{align*}

\subsubsection*{Equilibrium Analysis}
To find the equilibria of this system, we set each equation to zero, i.e. $f_1 = f_2 = f_3 = f_4 = 0$.
\begin{align*}
	f_3 = 0 \implies N = 0 \ \ \text{ or }\ \  \overline{m} - \overline{n} = \theta
\end{align*}
Let's consider the non-trivial solution, and thus $\overline{m} - \overline{n} = \theta$.  Then $\overline{a}(\overline{m}, \overline{n}) = \dfrac{\alpha\tau}{\sqrt{A}}$ where $A = \sigma^2 + \beta^2 + \tau^2$.
\begin{align*}
	f_1 = 0 \implies M = 0 \ \ \text{ or }\ \  N = \frac{d}{e\overline{a}(\overline{m}, \overline{n})} = \frac{d\sqrt{A}}{e\alpha\tau}
\end{align*}
Again, let's consider the non-trivial solution, and thus $\dfrac{d\sqrt{A}}{e\alpha\tau}$.
\begin{align*}
	f_4 = 0 \implies N = K \ \ \text{ or }\ \ \overline{n} = \phi
\end{align*}
Since $M \neq 0$, then $f_2 = 0 \implies N \neq K$.  Thus $\overline{n} = \phi$, and thus $\overline{m} = \theta + \phi$ and $\overline{r}(\overline{n}) = \dfrac{\rho \gamma}{\sqrt{B}}$ where $ B = \beta^2 + \gamma^2 $.
\begin{align*}
	f_2 = 0 \implies M = \frac{\rho \gamma \sqrt{A}}{\alpha\tau\sqrt{B}}\left(1 - \frac{d\sqrt{A}}{e\alpha\tau K}\right)
\end{align*}
So the coexistence equilibrium solution is
\begin{align*}
	C^* = (M^*, N^*, \overline{m}^*, \overline{n}^*) = \left(\frac{\rho\gamma\sqrt{A}}{\alpha\tau\sqrt{B}}\left(1 - \frac{d\sqrt{A}}{e\alpha\tau K}\right)\ ,\ \frac{d\sqrt{A}}{e\alpha\tau}\ ,\ \theta + \phi\ ,\ \phi\right)
\end{align*}
To check for local stability, we find eigenvalues of the Jacobian of the system evaluated at $C^*$.
\begin{align*}
	&\begin{cases}
		\dfrac{\partial f_1}{\partial M} &= e\overline{a}(\overline{m}, \overline{n})N - d \\[.25cm]
		\dfrac{\partial f_1}{\partial N} &= e\overline{a}(\overline{m}, \overline{n})M \\[.25cm]
		\dfrac{\partial f_1}{\partial \overline{m}} &= \dfrac{MNe(\theta - (\overline{m} - \overline{n}))}{A}\overline{a}(\overline{m} , \overline{n}) \\[.25cm]
		\dfrac{\partial f_1}{\partial \overline{n}} &= \dfrac{MNe((\overline{m} - \overline{n}) - \theta)}{A}\overline{a}(\overline{m} , \overline{n})
	\end{cases}\\
	&\begin{cases}
		\dfrac{\partial f_2}{\partial M} &= -\overline{a}(\overline{m}, \overline{n})N \\[.25cm]
		\dfrac{\partial f_2}{\partial N} &= \overline{r}(\overline{n})\left(1 - \dfrac{2N}{K}\right) - \overline{a}(\overline{m}, \overline{n})M \\[.25cm]
		\dfrac{\partial f_2}{\partial \overline{m}} &= -\dfrac{MN(\theta - (\overline{m} - \overline{n}))}{A}\overline{a}(\overline{m}, \overline{n}) \\[.25cm]
		\dfrac{\partial f_2}{\partial \overline{n}} &= N\left[\dfrac{\phi - \overline{n}}{B}\overline{r}(\overline{n})\left(1 - \dfrac{N}{K}\right) - \dfrac{M((\overline{m} - \overline{n}) - \theta)}{A}\overline{a}(\overline{m}, \overline{n})\right]
	\end{cases}\\
	&\begin{cases}
		\dfrac{\partial f_3}{\partial M} &= 0 \\[.25cm]
		\dfrac{\partial f_3}{\partial N} &= \dfrac{\sigma_G^2e \overline{a}(\overline{m},\overline{n})(\theta-(\overline{m}-\overline{n})}{A} \\[.25cm]
		\dfrac{\partial f_3}{\partial \overline{m}} &= -\dfrac{\sigma_G^2e \overline{a}(\overline{m},\overline{n})N}{A} \left[{1 - \dfrac{(\overline{m}-\overline{n}-\theta)^2}{A}}\right] \\[.25cm]
		\dfrac{\partial f_3}{\partial \overline{n}} &= \dfrac{\sigma_G^2e \overline{a}(\overline{m},\overline{n})N}{A} \left[{1 - \dfrac{(\overline{m}-\overline{n}-\theta)^2}{A}}\right]
	\end{cases}\\
	&\begin{cases}
		\dfrac{\partial f_4}{\partial M} &= \dfrac{\beta_G^2\overline{a}(\overline{m},\overline{n})(\theta-(\overline{m}-\overline{n}))}{A} \\[.25cm]
		\dfrac{\partial f_4}{\partial N} &= - \dfrac{\beta_G^2\overline{r}(\overline{n})(\phi - \overline{n})}{kB} \\[.25cm]
		\dfrac{\partial f_4}{\partial \overline{m}} &= - \dfrac{\beta_G^2\overline{a}(\overline{m},\overline{n})M}{A} \left[{1 - \dfrac{(\overline{m}-\overline{n}-\theta)^2}{A}}\right] \\[.25cm]
		\dfrac{\partial f_4}{\partial \overline{n}} &= \dfrac{\beta_G^2\overline{r}(\overline{n})\left({1-\dfrac{N}{K}}\right)}{B}\left[{\dfrac{(\phi-\overline{n})^2}{B}-1}\right] + \dfrac{\beta_G^2\overline{a}(\overline{m},\overline{n})M}{A}\left[{1-\dfrac{(\overline{m}-\overline{n}-\theta)^2}{A}}\right]
	\end{cases}
\end{align*}


\subsection*{Special Case: $M^*=\dfrac{\rho\gamma\sqrt{A}}{\alpha\tau\sqrt{B}}\left({1-\dfrac{d\sqrt{A}}{Ke\alpha\tau}}\right), N=\dfrac{d\sqrt{A}}{Ke\alpha\tau}$}
$E^* = (\dfrac{\rho\gamma\sqrt{A}}{\alpha\tau\sqrt{B}}\left({1-\dfrac{d\sqrt{A}}{Ke\alpha\tau}}\right), \dfrac{d\sqrt{A}}{Ke\alpha\tau}, \theta+\phi, \phi)$. Then $ J^* = J\big|_{E^*} = $
\begin{align*}
\begin{small}
	\left(
	\begin{array}{cccc}
		0 & \dfrac{e\rho\gamma}{\sqrt{B}}\left({1-\dfrac{d\sqrt{A}}{Ke\alpha\tau}}\right) & 0 & 0 \\[.4cm]
		-\dfrac{d}{e} & -\dfrac{d\rho\gamma\sqrt{A}}{Ke\alpha\tau\sqrt{B}} & 0 & 0 \\[0.4cm]
		0 & 0 & -\dfrac{\sigma_G^2d}{A} & \dfrac{\sigma_G^2d}{A} \\[0.4cm]
		0 & 0 & -\dfrac{\beta_G^2\rho\gamma\left({1-\dfrac{d\sqrt{A}}{Ke\alpha\tau}}\right)}{A\sqrt{B}} & \dfrac{\beta_G^2\rho\gamma\left({1-\dfrac{d\sqrt{A}}{Ke\alpha\tau}}\right)}{\sqrt{B}}\left[{\dfrac{1}{A}-\dfrac{1}{B}}\right] \\[0.4cm]
	\end{array}
	\right)
\end{small}
\end{align*}

The characteristic polynomial is
\begin{align*}
	P(\lambda) &= \left|{\lambda I - J^*}\right| =
\begin{small}
\left|
	\begin{array}{cccc}
		\lambda & \dfrac{e\rho\gamma}{\sqrt{B}}\left({1-\dfrac{d\sqrt{A}}{Ke\alpha\tau}}\right) & 0 & 0 \\[.4cm]
		-\dfrac{d}{e} & \lambda+\dfrac{d\rho\gamma\sqrt{A}}{Ke\alpha\tau\sqrt{B}} & 0 & 0 \\[0.4cm]
		0 & 0 & \lambda+\dfrac{\sigma_G^2d}{A} & \dfrac{\sigma_G^2d}{A} \\[0.4cm]
		0 & 0 & -\dfrac{\beta_G^2\rho\gamma\left({1-\dfrac{d\sqrt{A}}{Ke\alpha\tau}}\right)}{A\sqrt{B}} & \lambda-\dfrac{\beta_G^2\rho\gamma\left({1-\dfrac{d\sqrt{A}}{Ke\alpha\tau}}\right)}{\sqrt{B}}\left[{\dfrac{1}{A}-\dfrac{1}{B}}\right] \\[0.4cm]
	\end{array}
	\right|
\end{small}
\end{align*}

Thus,
\begin{align*}
	P(\lambda) &= \left|
	\begin{array}{cc}
		\lambda & \dfrac{e\rho\gamma}{\sqrt{B}}\left({1-\dfrac{d\sqrt{A}}{Ke\alpha\tau}}\right) \\[.4cm]
		-\dfrac{d}{e} & \lambda+\dfrac{d\rho\gamma\sqrt{A}}{Ke\alpha\tau\sqrt{B}} \\[0.4cm]
	\end{array}
	\right| \cdot \left|
	\begin{array}{cc}
		\lambda+\dfrac{\sigma_G^2d}{A} & \dfrac{\sigma_G^2d}{A} \\[0.4cm]
-\dfrac{\beta_G^2\rho\gamma\left({1-\dfrac{d\sqrt{A}}{Ke\alpha\tau}}\right)}{A\sqrt{B}} & \lambda-\dfrac{\beta_G^2\rho\gamma\left({1-\dfrac{d\sqrt{A}}{Ke\alpha\tau}}\right)}{\sqrt{B}}\left[{\dfrac{1}{A}-\dfrac{1}{B}}\right] \\[0.4cm]
	\end{array}
	\right| \\[.2cm]
	&= P_1(\lambda)\cdot P_2(\lambda)
\end{align*}

Thus the zeros of $P(\lambda)$ are the zeros of both $P_1(\lambda)$ and $P_2(\lambda)$.
\begin{align*}
	P_1(\lambda) &= \lambda^2 + \lambda\dfrac{d\rho\gamma\sqrt{A}}{Ke\alpha\tau\sqrt{B}} + \dfrac{d\rho\gamma}{\sqrt{B}}\left({1-\dfrac{d\sqrt{A}}{Ke\alpha\tau}}\right) \\
	\implies \lambda_{1,2} &= \frac{1}{2}\left[-\dfrac{d\rho\gamma\sqrt{A}}{Ke\alpha\tau\sqrt{B}} \pm \sqrt{\Delta}\right]
\end{align*}

Where $\Delta = \left({\dfrac{d\rho\gamma\sqrt{A}}{Ke\alpha\tau\sqrt{B}}}\right)^2 - \dfrac{4d\rho\gamma}{\sqrt{B}}\left(1 - \dfrac{d\sqrt{A}}{Ke\alpha\tau}\right)$.  Since $N^* = \dfrac{d\sqrt{A}}{e\alpha\tau} < K$, $\sqrt{\Delta} < \left|{\dfrac{d\rho\gamma\sqrt{A}}{Ke\alpha\tau\sqrt{B}}}\right|$.  Thus $\text{Re} (\lambda_{1,2}) < 0 \iff \left(\dfrac{d\rho\gamma\sqrt{A}}{Ke\alpha\tau\sqrt{B}}\right)^2 > \dfrac{4d\rho\gamma}{\sqrt{B}} \left({1-\dfrac{d\sqrt{A}}{Ke\alpha\tau}}\right)$.  So the coexistence equilibrium is stable if 
\begin{align*}
	\left(\dfrac{d\rho\gamma\sqrt{A}}{Ke\alpha\tau\sqrt{B}}\right)^2 > \dfrac{4d\rho\gamma}{\sqrt{B}} \left({1-\dfrac{d\sqrt{A}}{Ke\alpha\tau}}\right)
\end{align*}

For simplicity, let $ C = \dfrac{d\sigma_G^2}{A}, \  \ D = \dfrac{\beta_G^2\rho\gamma\left({1-\dfrac{d\sqrt{A}}{Ke\alpha\tau}}\right)}{A\sqrt{B}}, \  \ E = \dfrac{\beta_G^2\rho\gamma\left({1-\dfrac{d\sqrt{A}}{Ke\alpha\tau}}\right)}{B^{3/2}} $

\begin{align*}
	P_2(\lambda) &= \lambda^2 + (C+E-D)\lambda + CE = 0 \\
	\implies \lambda_{3,4} &= \frac{1}{2}\left[{-(C+E-D) \pm \sqrt{\Delta}}\right]
\end{align*}

Where
\begin{align*}
	\Delta = (C+E-D)^2 - 4CE
\end{align*} 

Again, since $N^*=\dfrac{d\sqrt{A}}{e\alpha\tau} < K$ and $C,D,E > 0 $, $\sqrt{\Delta} < \left|{C+E-D}\right|$. Thus $\text{Re} (\lambda_{3,4}) < 0 \iff C+E>D \text{ and } C^2 + D^2 + E^2 > 2(CD + DE + CE)$.  So the coexistence equilibrium is stable if 
\begin{align*}
	C+E>D \\
	C^2 + D^2 + E^2 > 2(CD + DE + CE)
\end{align*}

\end{document}