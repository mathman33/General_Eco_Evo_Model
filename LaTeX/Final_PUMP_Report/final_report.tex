%------------------------------------------------------------------------------
% Beginning of journal.tex
%------------------------------------------------------------------------------
%
% AMS-LaTeX version 2 sample file for journals, based on amsart.cls.
%
%        ***     DO NOT USE THIS FILE AS A STARTER.      ***
%        ***  USE THE JOURNAL-SPECIFIC *.TEMPLATE FILE.  ***
%
% Replace amsart by the documentclass for the target journal, e.g., tran-l.
%
\documentclass{amsart}

%     If your article includes graphics, uncomment this command.
\usepackage{graphicx}
\usepackage{array}
\newcolumntype{L}[1]{>{\raggedright\let\newline\\\arraybackslash\hspace{0pt}}m{#1}}
\newcolumntype{C}[1]{>{\centering\let\newline\\\arraybackslash\hspace{0pt}}m{#1}}
\newcolumntype{R}[1]{>{\raggedleft\let\newline\\\arraybackslash\hspace{0pt}}m{#1}}

\usepackage[hang]{subfigure}
\usepackage{placeins}

\newtheorem{theorem}{Theorem}[section]
\newtheorem{lemma}[theorem]{Lemma}

\theoremstyle{definition}
\newtheorem{definition}[theorem]{Definition}
\newtheorem{example}[theorem]{Example}
\newtheorem{xca}[theorem]{Exercise}

\theoremstyle{remark}
\newtheorem{remark}[theorem]{Remark}

\numberwithin{equation}{section}

%    Absolute value notation
\newcommand{\abs}[1]{\lvert#1\rvert}

%    Blank box placeholder for figures (to avoid requiring any
%    particular graphics capabilities for printing this document).
\newcommand{\blankbox}[2]{%
  \parbox{\columnwidth}{\centering
%    Set fboxsep to 0 so that the actual size of the box will match the
%    given measurements more closely.
    \setlength{\fboxsep}{0pt}%
    \fbox{\raisebox{0pt}[#2]{\hspace{#1}}}%
  }%
}
\allowdisplaybreaks



\begin{document}

\title[]{The Effects of Intraspecific Genetic Variation on the Dynamics of Predator-Prey Ecological Communities\textsuperscript{\tiny *}}

%    Information for first author
\author{Samuel R. Fleischer}
\email{samuel.fleischer.746@my.csun.edu}

\author{Pablo Chavarria}
\email{pablo.chavarria.189@my.csun.edu}

%    Address of record for the research reported here
\address{Department of Mathematics, California State University, Northridge}
%    Current address
\curraddr{}
%    \thanks will become a 1st page footnote.
\thanks{\textsuperscript{\tiny *}Partially supported by the National Science Foundation grant DMS-1247679}
\date{\today}

%    Information for second author
% \author{Pablo Chavarria}
% \address{Department of Mathematics, California State University, Northridge}
% \email{Pablo.Chavarria.189@my.csun.edu}
% \thanks{}

%    General info
% \subjclass[2000]{Primary 54C40, 14E20; Secondary 46E25, 20C20}


% \dedicatory{This paper is dedicated to our advisors.}

\keywords{Mathematical Evolutionary Ecology, Eco-Evo Feedback, Coevolution}


















\begin{abstract}
Predator-prey interactions are ubiquitous in nature and have captured the attention of ecologists and mathematicians.  Previous studies have focused on coexistence dynamics without taking into account phenotypic and genetic variation within a species.  Recent ecological models have incorporated evolutionary variables in order to further understand predator/prey and competitive dynamics.  General classifications of possible dynamics exist, but no previous model has provided enough flexibility to generate all dynamics.  We formulate new models for coevolution in generalized ditrophic predator-prey systems by incorporating quantitative characters relevant to predation in both prey and predator.  We study the impact of such trait variation by means of theoretical analysis and numerical simulations.
\end{abstract}

\maketitle





























\section{Introduction}

Trait variation plays an important role in understanding the variety of dynamics of predator-prey interactions seen in nature.  Evidence shows that species evolve over time but this is ignored in most classical mathematical ecological models.  Models for purely ecological predator-prey interactions with different functional responses are well understood to some extent.  Several recent studies have incorporated trait variation into these traditional ecological models in a variety of ways.  All of them have demonstrated much richer dynamics compared to the classical models.

In particular, Abrams and Matsuda introduced vulnerability as an evolutionary variable for a prey species.  This model results in chaotic, cyclic, and stable dynamics under various conditions \cite{Abrams_Matsuda_1997}.  Saloniemi introduced quantitative traits in a coevolutionary model.  Attack rate was defined as a linear function of both predator and prey trait values.  This model also produces chaotic dynamics under certain conditions \cite{Saloniemi_1993}.  More recently, Schreiber et. al. proposed an apparent competition model with Gaussian attack rate functions for an evolving generalist predator on two non-evolving prey populations.  In contrast with classical apparent competition theory, this model provided evidence that apparent competition can give rise to stable facilitation between under certain conditions \cite{Schreiber_2011}.

The variety of dynamics produced by incorporating evolutionary variables into purely ecological models is both ecologically and mathematically relevant.  The modified models and results have drawn attention from theoretical ecologists and applied mathematicians.  Our work was motivated by both Schreiber et. al. and Saloniemi.  In particular, we combine the coevolutionary dynamics of Saloniemi's work with the Gaussian attack rate and density-dependent prey growth rate from Schreiber et. al. \cite{Schreiber_2011}, \cite{Saloniemi_1993}.

We build a general ditrophic model of $u$ predator populations and $v$ prey populations by assuming
\begin{itemize}
	\item[-] predator exponential decay in the absence of prey
	\item[-] prey logicistic density-dependent growth in the absence of predator
	\item[-] linear functional response
	\item[-] all ecological parameters are dependent on evoloving genetic quantitative traits of both predator and prey
\end{itemize}

We then provide two specific manifestations of the general model.  In the first model, we assume the attack rate is a Gaussian function of both their trait values.  The second model is an expansion of the first model; we incorporate stabilizing selection in the form of a Gaussian growth rate function of the prey trait value.

Section 2 describes the formulation of the general ditrophic model, and the two specific models described above.  Section 3 describes the analytical results of the reduction of the two specific models to one predator and one prey.  Section 4 summarizes the work, and section 5 discusses its implications and future work.


%------------------------------------------------

































\section{Models and Methods}

Consider the ditrophic dynamics of $u$ predator populations with densities $M_i = M_i(t)$, consuming $v$ prey populations with densities $N_j = N_j(t)$, respectively ($i = 1, \dots, u$, and $j = 1, \dots, v$).  Ecological parameters will be defined as functions of predator phenotypic values, $m_i$, and prey phenotypic values, $n_j$, of quantitative traits.  We assume these traits can be measured in the same unit, or can be transformed into the same unit \cite{Saloniemi_1993}.  Assume predator traits are normally distributed with mean $\overline{m_i} = \overline{m_i}(t)$ and constant variances $\sigma_i^2$, and prey traits are normally distributed with mean $\overline{n_j} = \overline{n_j}(t)$ and constant variances $\beta_j^2$, i.e., their distributions are given by
\begin{equation}
	\label{distributions}
	\begin{aligned}
		p(m_i, \overline{m_i}) &= \frac{1}{\sqrt{2\pi\sigma_i^2}}\exp\left[-\frac{(m_i - \overline{m_i})^2}{2\sigma_i^2}\right] \\
		p(n_j, \overline{n_j}) &= \frac{1}{\sqrt{2\pi\beta_j^2}}\exp\left[-\frac{(n_j - \overline{n_j})^2}{2\beta_j^2}\right]
	\end{aligned}
\end{equation}
These variances have genetic and environmental components:
\begin{equation}
	\label{variances}
	\begin{aligned}
		\sigma_i^2 = \sigma_{Gi}^2 + \sigma_{Ei}^2 \\
		\beta_i^2 = \beta_{Gi}^2 + \beta_{Ei}^2
	\end{aligned}
\end{equation}

Assuming predator $i$ has a linear functional response with attack rate $a_{ij} = a_{ij}(m_i, n_j)$ on prey $j$, converts consumed prey $j$ into offspring with efficiencies $e_{ij} = e_{ij}(m_i, n_j)$, and experiences a per-capita mortality rate $d_i(m_i)$, then the fitness of a predator with phenotype $m_i$ is
\begin{equation}
	\label{predator_fitness}
	\begin{aligned}
		W_i(N_1, \dots, &N_u, M_i, n_1, \dots, n_v, m_i) \\
		&= \sum\limits_{j = 1}^{v}\left[e_{ij}(m_i, n_j)a_{ij}(m_i, n_j)N_i\right] - d_i(m_i)
	\end{aligned}
\end{equation}
and the mean fitness of the $i$\textsuperscript{th} predator population is
\begin{equation}
	\label{avg_predator_fitness}
	\begin{aligned}
		\overline{W_i}(N_1, \dots, &N_u, M_i, \overline{n_1}, \dots, \overline{n_v}, \overline{m_i}) \\
		&= \int\limits_{\mathbb{R}^{v+1}}^{}W_ip(m_i, \overline{m_i})\prod\limits_{j = 1}^{v}p(n_j, \overline{n_j})dm_i\prod\limits_{j = 1}^{v}dn_j
	\end{aligned}
\end{equation}

Assume in the absence of any predators, each prey species experiences logistic-type growth with growth rates $r_j = r_j(n_j)$ and carrying capacities $K_j = K_j(n_j)$.  Under these assumptions, the fitness of a prey individual with phenotype $n_j$ is
\begin{equation}
	\label{prey_fitness}
	\begin{aligned}
		Y_j(N_j, M_1, \dots, &M_v, n_j, m_1, \dots, m_v) \\
		&= r_j(n_j)\left(1 - \frac{N_j}{K_j(n_j)}\right) - \sum\limits_{i = 1}^{u}\left[a_{ij}(m_i, n_j)M_i\right]
	\end{aligned}
\end{equation}
and the mean fitness of the $j$\textsuperscript{th} prey population is
\begin{equation}
	\label{avg_prey_fitness}
	\begin{aligned}
		\overline{Y_j}(N_j, M_1, \dots, &M_v, \overline{n_j}, \overline{m_1}, \dots, \overline{m_v}) \\
		&= \int\limits_{\mathbb{R}^{v+1}}^{}Y_jp(n_j, \overline{n_j})\prod\limits_{i = 1}^{u}p(m_i, \overline{m_i})dn_j\prod\limits_{i = 1}^{u}dm_i
	\end{aligned}
\end{equation}
Thus, the ecological components of this model are
\begin{align*}
	\frac{dM_i}{dt} &= M_i\overline{W_i} \\[5px]
	\frac{dN_j}{dt} &= N_j\overline{Y_j}
\end{align*}

Assuming the evolutionary variables stay normally distributed, then their evolution is proportional to the partial derivative of the fitness function with respect to that variable.  In other words, evolution is always in the direction which increases the mean fitness of the population \cite{Lande_1976}.  This gives rise to the evolutionary components of this model:
\begin{align*}
	\frac{d\overline{m_i}}{dt} &= \sigma_{Gi}^2\frac{\partial \overline{W_i}}{\partial \overline{m_i}} \\[5px]
	\frac{d\overline{n_j}}{dt} &= \beta_{Gj}^2\frac{\partial \overline{Y_j}}{\partial \overline{n_j}}
\end{align*}
The complete ditrophic model of $u$ predator species and $v$ prey species is given by
\begin{subequations}
	\label{general_model}
	\begin{align}
		\label{eq:general_model_a}
		\frac{dM_i}{dt} &= M_i\overline{W_i} \\[5px]
		\label{eq:general_model_b}
		\frac{dN_j}{dt} &= N_j\overline{Y_j} \\[5px]
		\label{eq:general_model_c}
		\frac{d\overline{m_i}}{dt} &= \sigma_{Gi}^2\frac{\partial \overline{W_i}}{\partial \overline{m_i}} \\[5px]
		\label{eq:general_model_d} 
		\frac{d\overline{n_j}}{dt} &= \beta_{Gj}^2\frac{\partial \overline{Y_j}}{\partial \overline{n_j}}
	\end{align}
\end{subequations}
where $i = 1, \dots, u$ and $j = 1, \dots, v$. \\

The general idea of this model is that all ecological parameters can be defined as functions of the evolutionary variables, which themselves change in proportion to the change in fitness.  This ecological-evolutionary feedback, ``Eco-Evo Feedback'', is what drives the vast array of dynamics produced.

This model is a branch of Khibnik and Kondrashov's General Model of Coevolution \cite{Khibnik_Kondrashov_1997}, which describes a multitrophic ecological system of species, each of which may have a number of quantitative traits which affect each species' fitness.

Since community genetic changes occur over many generations, evolutionary variables are considered to be ``slow'' variables.  Their speeds are parameters in model given by $\sigma_{Gi}^2$ and $\beta_{Gi}^2$; they are the constants of proportionality between the change in the trait value and the change in fitness.  In contrast, ecological changes may happen within a single generation, and so the ecological variables are considered to be ``fast'' variables.  These differing timescales allow us to consider the coevolution models as two separate subsystems, the ecological and the evolutionary.  In the context of the ecological subsystem, the evolutionary variables can be viewed as slowly changing parameters.

In the following two sections we provide two particular manifestations of (\ref{general_model}).  In the first model, all ecological parameters are constant except the attack rate, which we define as a function of the evolutionary variables.  The second is an extension of the first, in which a prey's growth rate is defined as a function of its evolutionary variable.





















\subsection{Model 1}

This first model is a coevolutionary extension of Schreiber's et. al. apparent competition model in which the predator is assumed to have a quantitative trait and evolves in response to the ecological changes of two {\it non-evolving} prey species \cite{Schreiber_2011}.  In contrast to Schreiber et.~al., we assume an individual predator's attack rate on an individual prey is dependent on both trait values, is maximal at an optimal trait {\it difference}, $m_i - n_j = \theta_{ij}$, and decreases away from this optimal trait difference in a Gaussian manner, i.e.,
\begin{equation}
	\label{attack_rate}
	a_{ij}(m_i, n_j) = \alpha_{ij}\exp{\left[-\frac{((m_i - n_j) - \theta_{ij})^2}{2\tau_{ij}^2}\right]}
\end{equation}
where $\alpha_{ij}$ is the maximal attack and $\tau_{ij}$ determines how steeply the attack rate declines with distance from the optimal trait difference $\theta_{ij}$.  In effect, $\tau_{ij}$ determines how phenotypically specialized predator $i$ must be to use prey $j$.  We assume the prey growth rates and carrying capacities, $r_j$ and $K_j$, and the predator death rates and efficiencies, $d_i$, and $e_{ij}$, are constant.  Under these assumptions, the average attack rate of predator species $i$ on prey species $j$ is
\begin{equation}
	\label{average_attack_rate}
	\begin{aligned}
		\overline{a_{ij}}(\overline{m_i}, \overline{n_j}) &= \int\limits_{\mathbb{R}^2}a_{ij}(m_i, n_j)p(m_i, \overline{m_i})p(n_j, \overline{n_j})dm_idn_j \\
		&= \frac{\alpha_{ij}\tau_{ij}}{\sqrt{A_{ij}}}\exp{\left[-\frac{((\overline{m_i} - \overline{n_j}) - \theta_{ij})^2}{2A_{ij}}\right]}
	\end{aligned}
\end{equation}
where $A_{ij} = \tau_{ij}^2 + \sigma_i^2 + \beta_j^2$.  (See the appendix for the derivation of (\ref{average_attack_rate})).  (\ref{avg_predator_fitness}), and (\ref{avg_prey_fitness}) now yield eplicit formulas for $\overline{W_i}$ and $\overline{Y_j}$ in terms of (\ref{average_attack_rate}):
\begin{equation}
	\label{model_1_avg_pred_fitness}
	\overline{W_i} = \sum\limits_{j = 1}^{v}\left[e_{ij}\overline{a_{ij}}(\overline{m_i}, \overline{n_j})N_i\right] - d_i
\end{equation}
\begin{equation}
	\label{model_1_avg_prey_fitness}
	\overline{Y_j} = r_j\left(1 - \frac{N_j}{K_j}\right) - \sum\limits_{i = 1}^{u}\left[\overline{a_{ij}}(\overline{m_i}, \overline{n_j})M_i\right]
\end{equation}

(See the appendix for the derivations of (\ref{model_1_avg_pred_fitness}) and (\ref{model_1_avg_prey_fitness})).  Relavent partial derivatives of (\ref{model_1_avg_pred_fitness}) and (\ref{model_1_avg_prey_fitness}) are easily computable:
\begin{equation}
	\label{model_1_pred_fitness_partial}
	\frac{\partial \overline{W_i}}{\partial \overline{m_i}} = \sum\limits_{j = 1}^{v}\left[\frac{e_{ij}N_j(\theta_{ij} - (\overline{m_i} - \overline{n_j}))}{A_{ij}}\overline{a_{ij}}(\overline{m_i}, \overline{n_j})\right]
\end{equation}
\begin{equation}
	\label{model_1_prey_fitness_partial}
	\frac{\partial \overline{Y_j}}{\partial \overline{n_j}} = \sum\limits_{i = 1}^{u}\left[\frac{M_i(\theta_{ij} - (\overline{m_i} - \overline{n_j}))}{A_{ij}}\overline{a_{ij}}(\overline{m_i}, \overline{n_j})\right]
\end{equation}
Thus (\ref{general_model}) simplifies:
\begin{subequations}
	\label{model1}
	\begin{align}
		\label{eq:model1_a}
		\frac{dM_i}{dt} &= M_i\left[\sum\limits_{j = 1}^{v}\left[e_{ij}\overline{a_{ij}}(\overline{m_i}, \overline{n_j})N_i\right] - d_i\right] \\[5px]
		\label{eq:model1_b}
		\frac{dN_j}{dt} &= N_j\left[r_j\left(1 - \frac{N_j}{K_j}\right) - \sum\limits_{i = 1}^{u}\left[\overline{a_{ij}}(\overline{m_i}, \overline{n_j})M_i\right]\right] \\[5px]
		\label{eq:model1_c}
		\frac{d\overline{m_i}}{dt} &= \sigma_{Gi}^2\sum\limits_{j = 1}^{v}\left[\frac{e_{ij}N_j(\theta_{ij} - (\overline{m_i} - \overline{n_j}))}{A_{ij}}\overline{a_{ij}}(\overline{m_i}, \overline{n_j})\right] \\[5px]
		\label{eq:model1_d}
		\frac{d\overline{n_j}}{dt} &= \beta_{Gj}^2\sum\limits_{i = 1}^{u}\left[\frac{M_i(\theta_{ij} - (\overline{m_i} - \overline{n_j}))}{A_{ij}}\overline{a_{ij}}(\overline{m_i}, \overline{n_j})\right]
	\end{align}
\end{subequations}
Refer to Table 1 for parameters and their contextual meanings.















\subsection{Model 2}

This second model introduces stabilizing selection to Model 1 by assuming each prey species has an optimal trait value by which growth rate is maximized, and decreases away from the optimal trait value in a Gaussian manner, i.e.
\begin{equation}
	\label{growth_rate}
	r_j(n_j) = \rho_j\exp{\left[-\frac{(n_j - \phi_j)^2}{2\gamma_j^2}\right]}
\end{equation}
where $\rho_j$ is the maximal growth rate of the $j$\textsuperscript{th} prey species and $\gamma_j$ determines how steeply the growth rate declines with distance from the optimal trait value $\phi_j$.  In effect, $\gamma_j$ determines how far prey $j$ can deviate from its optimal trait value while still maintaining an adequate growth rate.  Under these assumptions, the average growth rate of prey species $j$ is
\begin{equation}
	\label{average_growth_rate}
	\begin{aligned}
		\overline{r_j}(\overline{n_j}) &= \int\limits_{\mathbb{R}}^{}r_j(n_j)p(n_j, \overline{n_j})dn_j \\
		&= \frac{\rho_j\gamma_j}{\sqrt{B_j}}\exp\left[-\frac{(\overline{n_j} - \phi_j)^2}{2B_j}\right]
	\end{aligned}
\end{equation}
where $B_j = \beta_j^2 + \gamma_j^2$.  (See the appendix for the derivation of (\ref{average_growth_rate})).  Since $\overline{W_i}$ is not dependent on $r_j$, (\ref{model_1_avg_pred_fitness}) suffices, but $\overline{Y_j}$ must be recalculated since it is dependent on $r_j$.  (\ref{avg_prey_fitness}) yields an explicit formula in terms of (\ref{average_attack_rate}) and (\ref{average_growth_rate}):
\begin{equation}
	\label{model_2_avg_prey_fitness}
	\overline{Y_j} = \overline{r_j}(\overline{n_j})\left(1 - \frac{N_j}{K_j}\right) - \sum\limits_{i = 1}^{u}\left[\overline{a_{ij}}(\overline{m_i}, \overline{n_j})M_i\right]
\end{equation}

(See the appendix for the derivation of (\ref{model_2_avg_prey_fitness})).  Since $\overline{W_i}$ did not change from Model 1, (\ref{eq:model1_c}) is sufficient for the right hand side of (\ref{eq:general_model_c}).  However, the right hand side of (\ref{eq:general_model_d}) must be recalculated.
\begin{equation}
	\label{model_2_prey_fitness_partial}
	\begin{aligned}
		\frac{\partial \overline{Y_j}}{\partial \overline{n_j}} = \overline{r_j}(\overline{n_j})\left(1 - \frac{N_j}{K_j}\right)\frac{\phi_j - \overline{n_j}}{B_j} + \sum\limits_{i = 1}^{u}\left[\frac{M_i(\theta_{ij} - (\overline{m_i} - \overline{n_j}))}{A_{ij}}\overline{a_{ij}}(\overline{m_i}, \overline{n_j})\right]
	\end{aligned}
\end{equation}
Thus (\ref{general_model}) simplifies:
\begin{subequations}
	\label{model2}
	\begin{align}
		\label{eq:model2_a}
		\frac{dM_i}{dt} &= M_i\left[\sum\limits_{j = 1}^{v}\left[e_{ij}\overline{a_{ij}}(\overline{m_i}, \overline{n_j})N_i\right] - d_i\right] \\[5px]
		\label{eq:model2_b}
		\frac{dN_j}{dt} &= N_j\left[\overline{r_j}(\overline{n_j})\left(1 - \frac{N_j}{K_j}\right) - \sum\limits_{i = 1}^{u}\left[\overline{a_{ij}}(\overline{m_i}, \overline{n_j})M_i\right]\right] \\[5px]
		\label{eq:model2_c}
		\frac{d\overline{m_i}}{dt} &= \sigma_{Gi}^2\sum\limits_{j = 1}^{v}\left[\frac{e_{ij}N_j(\theta_{ij} - (\overline{m_i} - \overline{n_j}))}{A_{ij}}\overline{a_{ij}}(\overline{m_i}, \overline{n_j})\right] \\[5px]
		\label{eq:model2_d}
		\begin{split}
			\frac{d\overline{n_j}}{dt} &= \beta_{Gj}^2\Bigg[\overline{r_j}(\overline{n_j})\left(1 - \frac{N_j}{K_j}\right)\frac{\phi_j - \overline{n_j}}{B_j} \\
			&\ \ \ \ \ \ \ + \sum\limits_{i = 1}^{u}\left[\frac{M_i(\theta_{ij} - (\overline{m_i} - \overline{n_j}))}{A_{ij}}\overline{a_{ij}}(\overline{m_i}, \overline{n_j})\right]\Bigg]
		\end{split}
	\end{align}
\end{subequations}
Refer to Tables 1 and 2 for parameters and their contextual meanings.

































\pagebreak
\section{Results}
\subsection{Pairwise Predator-Prey Dynamics of Model 1}
If there is only one predator species and one prey species, then (\ref{model1}) simplifies:
\begin{subequations}
	\label{MODEL1}
	\begin{align}
		\label{eq:MODEL1_A}
		\frac{dM}{dt} &= M\left[e\overline{a}(\overline{m}, \overline{n})N - d\right] \\[5px]
		\label{eq:MODEL1_B}
		\frac{dN}{dt} &= N\left[r\left(1 - \frac{N}{K}\right) - \overline{a}(\overline{m}, \overline{n})M\right] \\[5px]
		\label{eq:MODEL1_C}
		\frac{d\overline{m}}{dt} &= \sigma_{G}^2\frac{eN(\theta - (\overline{m} - \overline{n}))}{A}\overline{a}(\overline{m}, \overline{n}) \\[5px]
		\label{eq:MODEL1_D}
		\frac{d\overline{n}}{dt} &= \beta_{G}^2\frac{M(\theta - (\overline{m} - \overline{n}))}{A}\overline{a}(\overline{m}, \overline{n})
	\end{align}
\end{subequations}

There are three classifications of equilibria of (\ref{MODEL1}): extinction, exclusion, and coexistence.  Although there are an infinite amount of equilibrium points for each of these three classifications, their {\it ecological} components are unique and are comparable with common ecological models.  Extinction equilibria are given by
\begin{equation}
	\label{extinction_MODEL1}
	(M^*, N^*, \overline{m}^*, \overline{n}^*) = (0, 0, \mu^*, \nu^*)
\end{equation}
where $\mu^*$ and $\nu^*$ are arbitrary values.  Exclusion equilibria are given by
\begin{equation}
	\label{exclusion_MODEL1}
	(M^*, N^*, \overline{m}^*, \overline{n}^*) = (0, K, \mu^* + \theta, \mu^*)
\end{equation}
where $\mu^*$ is an arbitrary value.  Coexistence equilibria are given by
\begin{equation}
	\label{coexistence_MODEL1}
	\begin{aligned}
		(M^*, N^*, \overline{m}^*, \overline{n}^*) = \left(\frac{r\sqrt{A}}{\alpha\tau}\left(1 - \frac{N^*}{K}\right), \frac{d\sqrt{A}}{e\alpha\tau}, \mu^* + \theta, \mu^*\right)
	\end{aligned}
\end{equation}
where $\mu^*$ is an arbitrary value.  Local stability analysis yields that all extinction equilibria are unstable, exclusion equilibria are locally asymptotically stable if
\begin{equation}
	\label{exclusion_stability_MODEL1}
	d > \frac{Ke\alpha\tau}{\sqrt{A}}
\end{equation}
and coexistence equilibria are locally asymptotically stable if
\begin{equation}
	\label{coexistence_stability_MODEL1}
	\frac{\sigma_G^2}{\beta_G^2} > \frac{r}{d}\left(1 - \frac{d\sqrt{A}}{Ke\alpha\tau}\right)
\end{equation}

(See the appendix for the derivation of the equilibria and local stability conditions.)  In nature, (\ref{exclusion_stability_MODEL1}) is consistent with the fact that exclusion is possible if the predator death rate is high.  Figure \ref{fig:constant_growth_exclusion} displays a simulation that results in stable exclusion.

Note that if (\ref{exclusion_stability_MODEL1}) holds then (\ref{coexistence_MODEL1}) is not biologically feasible ($M^* < 0$), implying that coexistent states do not exist, and so even though (\ref{coexistence_stability_MODEL1}) would hold (since all parameters are assumed to be positive), it would be irrelevant.  Since $\sigma_G^2/\beta_G^2$ is the ratio of predator and prey ``speeds'' of evolution, then intuitively, coexistence is stable if the predator is ``fast'' enough at evolving in comparison to the prey.  If this happens, the predator trait value ``catches up'' to the prey trait value.  Figure \ref{fig:constant_growth_coexistence_equilibrium} displays a simulation that results in stable coexistence.

Since (\ref{exclusion_stability_MODEL1}) and (\ref{coexistence_stability_MODEL1}) are not equal and opposite conditions, the possibility remains that neither conditon holds.  A natural question is what dynamics appear in these circumstances.  Numerical simulations provide insight into these dynamics.  Figure \ref{fig:constant_growth_arms_race_coexistence} depicts an evolutionary ``arms race'' between the predator and prey.  The prey has no particular optimal value, and the predator is not fast enough at evolving to catch up to the prey, so they continuously evolve in a linear fashion.  The ``arms race'' is due to that fact that there is no stabilizing selection in this model; the prey species has no reason to stop evolving, and the predator species has no reason to stop chasing the prey.  The next model shows one of many possible ways to introduce stabilizing selection for the purpose of avoiding an ``arms race''.



































\subsection{Pairwise Predator-Prey Dynamics of Model 2}
If there is only one predator species and one prey species, then (\ref{model2}) simplifies:
\begin{subequations}
	\label{MODEL2}
	\begin{align}
		\label{eq:MODEL2_A}
		\frac{dM}{dt} &= M\left[e\overline{a}(\overline{m}, \overline{n})N - d\right] \\[5px]
		\label{eq:MODEL2_B}
		\frac{dN}{dt} &= N\left[\overline{r}(\overline{n})\left(1 - \frac{N}{K}\right) - \overline{a}(\overline{m}, \overline{n})M\right] \\[5px]
		\label{eq:MODEL2_C}
		\frac{d\overline{m}}{dt} &= \sigma_{G}^2\frac{eN(\theta - (\overline{m} - \overline{n}))}{A}\overline{a}(\overline{m}, \overline{n}) \\[5px]
		\label{eq:MODEL2_D}
		\frac{d\overline{n}}{dt} &= \beta_{G}^2\Bigg[\overline{r}(\overline{n})\left(1 - \frac{N}{K}\right)\frac{\phi - \overline{n}}{B} + \frac{M(\theta - (\overline{m} - \overline{n}))}{A}\overline{a}(\overline{m}, \overline{n})\Bigg]
	\end{align}
\end{subequations}

Similarly to (\ref{MODEL1}), there are three classifications of equilibrium of system (\ref{MODEL2}): extinction, exclusion, and coexistence.  There are an infinite amount of equilibrium points for the extinction and exclusion classifications, but stabilizing selection provides a unique coexistence equilibrium point.  Extinction equilibria are given by (\ref{extinction_MODEL1}), and exclusion equilibria are given by (\ref{exclusion_MODEL1}).  The coexistence equilibrium point is given by
\begin{equation}
	\label{coexistence_MODEL2}
	\begin{aligned}
		(M^*, N^*, \overline{m}^*, \overline{n}^*) = \left(\frac{\rho\gamma\sqrt{A}}{\alpha\tau\sqrt{B}}\left(1 - \frac{N^*}{K}\right), \frac{d\sqrt{A}}{e\alpha\tau}, \phi + \theta, \phi\right)
	\end{aligned}
\end{equation}

Local stability analysis for extinction and exclusion equilibria is nearly identical to Model 1 - all extinction equilibria are unstable and exclusion equilibria are asymptotically stable if (\ref{exclusion_stability_MODEL1}) holds.  The coexistence equilibrium is asymptotically stable if
\begin{equation}
	\label{coexistence_stability_MODEL2}
	\frac{\sigma_G^2}{\beta_G^2} > \frac{\rho\gamma}{d\sqrt{B}}\left(1 - \frac{d\sqrt{A}}{Ke\alpha\tau}\right)\left(1 - \frac{A}{B}\right)
\end{equation}

(See the appendix for the derivation of the equilibria and local stability conditions.)  Similar to Model 1, exclusion is stable if the predator death rate is high enough, and if (\ref{exclusion_stability_MODEL1}) holds then (\ref{coexistence_MODEL2}) is not biologically feasible ($M^* < 0$), and so even though (\ref{coexistence_stability_MODEL2}) may hold, it would be irrelevant.  Again, since $\sigma_G^2/\beta_G^2$ is the ratio of predator and prey ``speeds'' of evolution, then coexistence is stable only if the predator is ``fast'' enough at evolving in comparison to the prey.  Figure \ref{fig:variable_growth_exclusion} displays a simulation that results in stable exclusion, and Figure \ref{fig:variable_growth_coexistence_equilibrium} displays a simulation that results in stable coexistence.

(\ref{exclusion_stability_MODEL1}) and (\ref{coexistence_stability_MODEL2}) are not equal and opposite conditions, so there is at least one type of non-equilibrium coexistence dynamic.  Numerical simulations provide insight into these dynamics.  Figure \ref{fig:variable_growth_stable_cycles} depicts long-term stable oscillatory behavior.  We can intuitively understand these dynamics by considering the inverse effects that the evolution of the prey trait has on its own fitness.  At the same time the prey evolves its own trait value away from the predator trait value (to minimize attack rate), it must also stay close enough to its optimal trait value $\phi_j$ to maintain an adequate growth rate.  These effects nullify each other whenever the prey trait value reaches a maximum or minimum.  Immediately after the prey trait value reverses direction, the prey has double incentive to evolve toward $\phi_j$: it increases its growth rate while minimizing the predator's attack rate.  Immediately after passing through $\phi_j$, however, the inverse effects take hold, and the cycle begins again.

Analytically, a Hopf Bifurcation occurs as parameters are changed in such a way that does not alter the dissatisfaction of (\ref{exclusion_stability_MODEL1}) but simply dissatisfies (\ref{coexistence_stability_MODEL2}).  This can happen under a number of conditions, as explained in the appendix.  Figure \ref{fig:contour_plot} shows one such condition, which compares $\frac{\sigma_G}{\beta_G}$ (the ratio of evolution ``speeds'' of predator and prey) vs. $\tau$ (the specialization constant).  The values seen on this plot are found by solving (\ref{coexistence_stability_MODEL2}) for zero, i.e.
\begin{equation}
	f\left(\frac{\sigma_G}{\beta_G}, \rho, \gamma, d, A, B, \alpha, \tau, e, K\right) = \frac{\sigma_G^2}{\beta_G^2} - \frac{\rho\gamma}{d\sqrt{B}}\left(1 - \frac{d\sqrt{A}}{Ke\alpha\tau}\right)\left(1 - \frac{A}{B}\right)
\end{equation}

This proves the existence of a limit cycle.  Note $f > 0$ implies stable coexistence but $f < 0$ does not imply stable exclusion.  A detailed discussion of the Hopf Bifurcation analysis will be investigated as one of the future works.





























\section{Summary}
The first part of section 2 describes the ditrophic model in general, without specifying the particular functions that govern the dynamics of the system.  The second part of section 2 describes two concrete manifestations of the general model by providing explicit functions for the ecological parameters.  Lastly, section 3 provides analytical results of the reduction of the two specific models to one predator and one prey.

The novelty of this model resides in the incorporation of coevolutionary dynamics with realistic and natural Gaussian functions and density-dependent prey growth rates.  In the first model, the attack rate is, in contrast to that of Schreiber et. al., dependent on both predator and prey trait values.  In the second model, prey growth rate is defined as a Gaussian function dependent on the prey's trait value.

The dynamics of the first model include the exclusion of the predator, a stable coexistence state, and an evolutionary ``arms race'' dynamic.  We note that in order to for the exclusion state to be stable, the predator death rate must be high enough, which matches with our biological intuition (see (\ref{exclusion_stability_MODEL1})).  For coexistence, we see that the predator must be ``fast'' enough at evolving in order for it to have any chance of survival.  The ``arms race'' coexistence dynamic is not found in classical ecological models, and can be interpreted as possible speciation.

The second model's dynamics include the exclusion of predator, and a coexistence state.  The conditions for stability of both exclusion and coexistence are parallel to that of the first model.  In contrast to the first model, however, the non-constant coexistence dynamic shows cyclic behavior.  The introduction of stabilizing selection for the prey prevented the evolutionary ``arms race''.  Instead, the second model provides long-term Red Queen Dynamics \cite{Khibnik_Kondrashov_1997} and possibly unique, globally stable limit cycles.
% Discuss the novelty of our work

% Summary of the outline of the paper

% Summary of the results for model 1

% Summary of the results for model 2
































\section{Discussion}

% Implications of the Model
	% Our model demonstrates a richer array of dynamics
Our model demonstrates that incorporating trait variation into ecological models is necessary in order to obtain the rich array of dynamics seen in nature.  Our work serves as further evidence that evolution plays a central role in determining the dynamics in predator-prey interactions.  Inclusion of evolution into more intricate and complex ecological models can provide insight into the true causes of exclusion and coexistence.

One of our immediate goals is to prove, for the second model, the existence of a globally stable unique limit cycle.  We would also like to expand our analytical results in general for more complex predator-prey interactions.  In particular, expanding Schreiber's et.~al.~apparent competition model by incorporating prey evolution ($1 \times 2$ system), exploring direct competition of two predator on one prey ($2 \times 1$ system), and possibly up to the general ditrophic $u \times v$ system.  Other expansions include multitrophic intraguid predation with coevolution, and prey resource competition.






























\FloatBarrier
\pagebreak
\section{Tables and Figures}

\begin{table}[h]
	\label{parameter_table_model_1}
    \begin{tabular}{||c|L{7cm}|L{2.5cm}||}\hline\hline
    {\bf Parameter} & {\bf Contextual Meaning} & {\bf Range of Biologically Meaningful Values} \\\hline\hline
    $r_j$ & intrinsic growth rate of prey $j$ & $(0, 1)$ \\\hline
    $K_j$ & carrying capacity of prey $j$ & $(1, \infty)$ \\\hline\hline
    $e_{ij}$ & efficiency of predator $i$ to turn prey $j$ into offspring & $(0, 0.5)$ \\\hline\hline
    $\alpha_{ij}$ & maximum value of the Gaussian attack rate function; maximum successful attack rate of predator $i$ on prey $j$ & $(0, 1)$ \\\hline
    $\tau_{ij}$   & variance of the Gaussian attack rate function; determines how specialized predator $i$ must be to use prey $j$ & $(0, 1)$ \\\hline
    $\theta_{ij}$ & mean value of the Gaussian attack rate function; trait difference that maximizes attack rate of predator $i$ and prey $j$; ``optimal'' difference with respect to the predator & $\mathbb{R}$ \\\hline\hline
    $\sigma_i$ & trait distribution variance of predator $i$; $\sigma_i^2 = \sigma_{Gi}^2 + \sigma_{Ei}^2$ & $\mathbb{R}^{+}$ \\\hline
    $\sigma_{Gi}$ & genetic portion of $\sigma_i$; determines the ``speed'' of evolution of predator $i$ & $(0, \sigma_i)$ \\\hline
    $\sigma_{Ei}$ & environmental portion of $\sigma_i$ & $(0, \sigma_i)$ \\\hline\hline
    $\beta_j$ & trait distribution variance of prey $j$; $\beta_j^2 = \beta_{Gj}^2 + \beta_{Ej}^2$ & $\mathbb{R}^{+}$ \\\hline
    $\beta_{Gj}$ & genetic portion of $\beta_j$; determines the ``speed'' of evolution of prey $j$ & $(0, \beta_j)$ \\\hline
    $\beta_{Ej}$ & environmental portion of $\beta_j$ & $(0, \beta_j)$ \\\hline\hline
    \end{tabular}
    \caption{\footnotesize{\bf Parameter Table - Model 1.}}
\end{table}
\begin{table}[h]
	\label{parameter_table_model_2}
    \begin{tabular}{||c|L{7cm}|L{2.5cm}||}\hline\hline
    {\bf Parameter} & {\bf Contextual Meaning} & {\bf Range of Biologically Meaningful Values} \\\hline\hline
    $\rho_j$ & maximum value of the Gaussian intrinsic growth rate function $j$ & $(0, 1)$ \\\hline
    $\gamma_j$ & variance of the Gaussian intrinsic growth rate function $j$; determines how dependent prey $j$ is on its optimal value $\phi_j$ & $\mathbb{R}^{+}$ \\\hline
    $\phi_j$ & mean value of the Gaussian intrinsic growth rate function $j$; ``optimal'' value with respect to the prey; maximizes the growth rate function & $\mathbb{R}$ \\\hline\hline
    \end{tabular}
    \caption{\footnotesize{\bf Parameter Table - Model 2.}}
\end{table}
\begin{centering}
	\begin{figure*}[h]
		\makebox[\linewidth][c]{%
			\centering
			\subfigure[Predator and Prey Densities vs. Time]{\label{fig:a}\includegraphics[width=0.7\textwidth]{figures/1x1/constant_growth/densities_exclusion.png}}%
			\subfigure[Predator and Prey Trait Values vs. Time]{\label{fig:b}\includegraphics[width=0.7\textwidth]{figures/1x1/constant_growth/traits_exclusion.png}}%
		}
		\caption{\footnotesize {\bf Model 1: Exclusion Equilibrium.} Parameters: $e = 0.05$, $d = 0.25$, $\alpha = 0.05$, $\theta = 0.1$, $\tau = 0.1$, $M_0 = 100$, $\overline{m}_0 = 0$, $\sigma = 0.25$, $\sigma_G = 0.2$, $K = 225$, $r = 0.2$, $N_0 = 120$, $\overline{n}_0 = 0.2$, $\beta = 0.25$, $\beta_G = 0.1$.  Under these conditions, the exclusion condition holds.}
		\label{fig:constant_growth_exclusion}
	\end{figure*}
	\begin{figure*}[h]
		\makebox[\linewidth][c]{%
			\centering
			\subfigure[Predator and Prey Densities vs. Time]{\label{fig:a}\includegraphics[width=0.7\textwidth]{figures/1x1/constant_growth/densities_stable_coexistence.png}}%
			\subfigure[Predator and Prey Trait Values vs. Time]{\label{fig:b}\includegraphics[width=0.7\textwidth]{figures/1x1/constant_growth/traits_stable_coexistence.png}}%
		}
		\caption{\footnotesize {\bf Model 1: Coexistence Equilibrium.} Parameters: $e = 0.2$, $d = 1$, $\alpha = 0.05$, $\theta = 0.1$, $\tau = 0.1$, $M_0 = 100$, $\overline{m}_0 = 0$, $\sigma = 0.25$, $\sigma_G = 0.2$, $K = 225$, $r = 0.2$, $N_0 = 120$, $\overline{n}_0 = 0.2$, $\beta = 0.25$, $\beta_G = 0.1$.  Under these conditions, the equilibrium condition holds.}
		\label{fig:constant_growth_coexistence_equilibrium}
	\end{figure*}
	\begin{figure*}[h]
		\makebox[\linewidth][c]{%
			\centering
			\subfigure[Predator and Prey Densities vs. Time]{\label{fig:a}\includegraphics[width=0.7\textwidth]{figures/1x1/constant_growth/densities_unstable_coexistence.png}}%
			\subfigure[Predator and Prey Trait Values vs. Time]{\label{fig:b}\includegraphics[width=0.7\textwidth]{figures/1x1/constant_growth/traits_unstable_coexistence.png}}%
		}
		\caption{\footnotesize {\bf Model 1: ``Arms Race'' Coexistence.} Parameters: $e = 0.5$, $d = 0.05$, $\alpha = 0.05$, $\theta = 0.1$, $\tau = 0.1$, $M_0 = 100$, $\overline{m}_0 = 0$, $\sigma = 0.25$, $\sigma_G = 0.22$, $K = 225$, $r = 0.2$, $N_0 = 120$, $\overline{n}_0 = 0.2$, $\beta = 0.25$, $\beta_G = 0.1$.  Under these condition, neither the coexistence of the exclusion stability criterion hold.}
		\label{fig:constant_growth_arms_race_coexistence}
	\end{figure*}

	\begin{figure*}[h]
		\makebox[\linewidth][c]{%
			\centering
			\subfigure[Predator and Prey Densities vs. Time]{\label{fig:a}\includegraphics[width=0.7\textwidth]{figures/1x1/variable_growth/stable_exclusion/densities.png}}%
			\subfigure[Predator and Prey Trait Values vs. Time]{\label{fig:b}\includegraphics[width=0.7\textwidth]{figures/1x1/variable_growth/stable_exclusion/traits.png}}%
		}
		\makebox[\linewidth][c]{%
			\centering
			\subfigure[Predator/Prey Density Phase Plane]{\label{fig:a}\includegraphics[width=0.7\textwidth]{figures/1x1/variable_growth/stable_exclusion/density_phase_plane.png}}%
			\subfigure[Predator/Prey Trait Value Phase Plane]{\label{fig:b}\includegraphics[width=0.7\textwidth]{figures/1x1/variable_growth/stable_exclusion/trait_phase_plane.png}}%
		}
		\caption{\footnotesize {\bf Model 2: Exclusion Equilibrium.} Parameters: $e = 0.05$, $d = 0.05$, $\alpha = 0.05$, $\theta = 0.1$, $\tau = 0.05$, $M_0 = 100$, $\overline{m}_0 = 1$, $\sigma = 0.25$, $\sigma_G = 0.1$, $K = 225$, $\rho = 0.5$, $\gamma = 0.3$, $\phi = 0.0$, $N_0 = 120$, $\overline{n}_0 = 1.0$, $\beta = 0.25$, $\beta_G = 0.1$.  Under these conditions, the exclusion criterion holds.}
		\label{fig:variable_growth_exclusion}
	\end{figure*}
	\begin{figure*}[h]
		\makebox[\linewidth][c]{%
			\centering
			\subfigure[Predator and Prey Densities vs. Time]{\label{fig:a}\includegraphics[width=0.7\textwidth]{figures/1x1/variable_growth/stable_coexistence/densities.png}}%
			\subfigure[Predator and Prey Trait Values vs. Time]{\label{fig:b}\includegraphics[width=0.7\textwidth]{figures/1x1/variable_growth/stable_coexistence/traits.png}}%
		}
		\makebox[\linewidth][c]{%
			\centering
			\subfigure[Predator/Prey Density Phase Plane]{\label{fig:a}\includegraphics[width=0.7\textwidth]{figures/1x1/variable_growth/stable_coexistence/density_phase_plane.png}}%
			\subfigure[Predator/Prey Trait Value Phase Plane]{\label{fig:b}\includegraphics[width=0.7\textwidth]{figures/1x1/variable_growth/stable_coexistence/trait_phase_plane.png}}%
		}
		\caption{\footnotesize {\bf Model 2: Coexistence Equilibrium.} Parameters: $e = 0.5$, $d = 0.05$, $\alpha = 0.05$, $\theta = 0.1$, $\tau = 0.1$, $M_0 = 1$, $\overline{m}_0 = 0$, $\sigma = 0.25$, $\sigma_G = 0.18$, $K = 225$, $\rho = 0.2$, $\gamma = 0.65$, $\phi = 0.0$, $N_0 = 120$, $\overline{n}_0 = 1$, $\beta = 0.25$, $\beta_G = 0.1$.  Under these conditions, the coexistence criterion holds.}
		\label{fig:variable_growth_coexistence_equilibrium}
	\end{figure*}

	\begin{figure*}[h]
		\makebox[\linewidth][c]{%
			\centering
			\subfigure[Predator and Prey Densities vs. Time]{\label{fig:a}\includegraphics[width=0.7\textwidth]{figures/1x1/variable_growth/stable_cycles/densities.png}}%
			\subfigure[Predator and Prey Trait Values vs. Time]{\label{fig:b}\includegraphics[width=0.7\textwidth]{figures/1x1/variable_growth/stable_cycles/traits.png}}%
		}
		\makebox[\linewidth][c]{%
			\centering
			\subfigure[Predator/Prey Density Phase Plane]{\label{fig:a}\includegraphics[width=0.7\textwidth]{figures/1x1/variable_growth/stable_cycles/density_phase_plane.png}}%
			\subfigure[Predator/Prey Trait Value Phase Plane]{\label{fig:b}\includegraphics[width=0.7\textwidth]{figures/1x1/variable_growth/stable_cycles/trait_phase_plane.png}}%
		}
		\caption{\footnotesize {\bf Model 2: Non-Equilibrium, Cyclic Coexistence.} Parameters: $e = 0.5$, $d = 0.05$, $\alpha = 0.05$, $\theta = 0.1$, $\tau = 0.1$, $M_0 = 1$, $\overline{m}_0 = 0$, $\sigma = 0.25$, $\sigma_G = 0.1$, $K = 225$, $\rho = 0.5$, $\gamma = 1.0$, $\phi = 0.0$, $N_0 = 120$, $\overline{n}_0 = 1$, $\beta = 0.25$, $\beta_G = 0.1$.  Under these conditions, neither the exclusion nor coexistence stability criterion hold.}
		\label{fig:variable_growth_stable_cycles}
	\end{figure*}
	\begin{figure*}[h]
		\makebox[\linewidth][c]{%
			\centering
			\subfigure[Predator and Prey Densities vs. Time]{\label{fig:a}\includegraphics[width=1.2\textwidth]{figures/1x1/variable_growth/contour_plots/tau_ratio_150417_125636.png}}%
		}
		\caption{\footnotesize {\bf Model 2: Contour Plot of Coexistence/Hopf Bifurcation Condition.} Parameter Values: $d = 0.05$, $\beta = 0.2$, $\rho = 0.5$, $\alpha = 0.1$, $e = 0.1$, $\sigma = 0.2$, $\gamma = 0.3$.  At higher values and very low values of $\tau$, the ratio of speeds of evolution ($\sigma_G/\beta_G$) is irrelevant to determining stable coexistence.  At intermediate values of $\tau$, higher ratios of speeds of evolution are required to have stable coexistence, and lower ratios result in cyclic coexistence or even stable exclusion.}
		\label{fig:contour_plot}
	\end{figure*}
	\begin{figure*}[h]
		\makebox[\linewidth][c]{%
			\centering
			\subfigure[Predator/Prey Density Phase Plane - $\sigma_G/\beta_G = 1.3$.]{\label{fig:a}\includegraphics[width=0.7\textwidth]{figures/1x1/variable_growth/contour_plots/density_phase_plane_limit_cycle.png}}%
			\subfigure[Predator/Prey Density Phase Plane - $\sigma_G/\beta_G = 1.5$.]{\label{fig:b}\includegraphics[width=0.7\textwidth]{figures/1x1/variable_growth/contour_plots/density_phase_plane_node.png}}%
		}
		\makebox[\linewidth][c]{%
			\centering
			\subfigure[Predator/Prey Trait Value Phase Plane - $\sigma_G/\beta_G = 1.3$.]{\label{fig:a}\includegraphics[width=0.7\textwidth]{figures/1x1/variable_growth/contour_plots/trait_phase_plane_limit_cycle.png}}%
			\subfigure[Predator/Prey Trait Value Phase Plane - $\sigma_G/\beta_G = 1.5$.]{\label{fig:b}\includegraphics[width=0.7\textwidth]{figures/1x1/variable_growth/contour_plots/trait_phase_plane_node.png}}%
		}
		\caption{\footnotesize {\bf Model 2: Non-Equilibrium, Cyclic Coexistence.} Parameter Values: $d = 0.05$, $\beta = 0.2$, $\rho = 0.5$, $\alpha = 0.1$, $e = 0.1$, $\sigma = 0.2$, $\gamma = 0.3$, $\tau = 0.05$.  Left phase-planes: $\sigma_G/\beta_G = 1.3$, stable limit cycle.  Right phase-planes: $\sigma_G/\beta_G = 1.5$, stable node.}
		\label{fig:hopf_bifurcation}
	\end{figure*}
\end{centering}




























\FloatBarrier
\pagebreak
\section*{Acknowledgements}
This work could not be completed without the help and guidance of our research advisor and mentor, Dr. Jing Li, and our biology consultant, Dr. Casey terHorst.  This work was conducted as part of the Pacific Math Alliance's CSU PUMP (Preparing Undergraduates through Mentoring towards PhDs) program, directed by Dr. Helena Noronha.  Samuel R. Fleischer and Pablo Chavarria were partially supported by the National Science Foundation grant DMS-1247679.






























\FloatBarrier
\pagebreak
\section*{Appendices}
\subsection*{Appendix 1: Derivation of Models 1 and 2}
\subsubsection*{\textbf{\textit{Derivation of (\ref{average_attack_rate})}}}
First note the following:
\begin{align*}
	\int\limits_{\mathbb{R}}\exp\left[-(ax^2 + bx + c)\right] = \sqrt{\frac{\pi}{a}}\exp\left[\frac{b^2}{4a} - c\right]
\end{align*}
\begin{align*}
	&\int\limits_{\mathbb{R}^2}a_{ij}(m_i, n_j)p(m_i, \overline{m_i})p(n_j, \overline{n_j})dm_idn_j \\
	&= \frac{\alpha_{ij}}{2\pi\sigma_i\beta_j}\int\limits_{\mathbb{R}^2}\exp\left[-\frac{((m_i - n_j) - \theta_{ij})^2}{2\tau_{ij}^2} - \frac{(m_i - \overline{m_i})^2}{2\sigma_i^2} - \frac{(n_j - \overline{n_j})^2}{2\beta_j^2}\right]dm_idn_j \\
	&= \frac{\alpha_{ij}}{2\pi\sigma_i\beta_j}\int\limits_{\mathbb{R}}\exp\left[-\frac{(n_j - \overline{n_j})^2}{2\beta_j^2}\right]\int\limits_{\mathbb{R}}\exp\left[-(am_i^2 + bm_i + c)\right]dm_idn_j
\end{align*}
where $a = \dfrac{\sigma_i^2 + \tau_{ij}^2}{2\sigma_i^2\tau_{ij}^2}$, $b = -\left(\dfrac{\sigma_i^2(n + \theta_{ij}) + \tau_{ij}^2\overline{m}}{\tau_{ij}^2\sigma_i^2}\right)$, $c = \dfrac{\sigma_i^2(n_j + \theta_{ij})^2 + \tau_{ij}^2\overline{m}^2}{2\sigma_i^2\tau_{ij}^2}$
\begin{align*}
	\implies \sqrt{\frac{\pi}{a}}\exp\left[\frac{b^2}{4a} - c\right] = \frac{\sigma_i\tau_{ij}\sqrt{2\pi}}{\sqrt{\sigma_i^2 + \tau_{ij}^2}}\exp\left[-\frac{((\overline{m} - n) - \theta_{ij})^2}{2(\sigma_i^2 + \tau_{ij}^2)}\right]
\end{align*}
Thus
\begin{align*}
	&\int\limits_{\mathbb{R}^2}a_{ij}(m_i, n_j)p(m_i, \overline{m_i})p(n_j, \overline{n_j})dm_idn_j \\
	&= \frac{\alpha_{ij}\tau_{ij}}{\beta_j\sqrt{2\pi}\sqrt{\sigma_i^2 + \tau_{ij^2}}}\int\limits_{\mathbb{R}}\exp\Bigg[-\frac{((\overline{m_i} - n_j) - \theta_{ij})^2}{2(\sigma_i^2 + \tau_{ij}^2)} - \frac{(n_j - \overline{n_j})^2}{2\beta_j^2}\Bigg]dn_j \\
	&= \frac{\alpha_{ij}\tau_{ij}}{\beta_j\sqrt{2\pi}\sqrt{\sigma_i^2 + \tau_{ij^2}}}\int\limits_{\mathbb{R}}\exp\left[-(am_i^2 + bm_i + c)\right]dn_j
\end{align*}
where $a = \dfrac{\tau_{ij}^2 + \sigma_i^2 + \beta_j^2}{2\beta_j^2(\sigma_i^2 + \tau_{ij}^2)}$, $b = -\dfrac{(\overline{m_i} - \theta_{ij})^2\beta_j^2 + (\sigma_i^2 + \tau_{ij}^2)\overline{n_j}}{\beta_j^2(\sigma_i^2 + \tau_{ij}^2)}$, and $c = \dfrac{(\overline{m_i} - \theta_{ij})^2\beta_j^2 + \overline{n}^2(\sigma_i^2 + \tau_{ij}^2)^2}{2\beta_j^2(\sigma_i^2 + \tau_{ij}^2)}$
\begin{align*}
	\implies \sqrt{\frac{\pi}{a}}&\exp\left[\frac{b^2}{4a} - c\right] = \frac{\beta_j\sqrt{2\pi}\sqrt{\sigma_i^2 + \tau_{ij}^2}}{\sqrt{\beta_j^2 + \sigma_i^2 + \tau_{ij}^2}}\exp\left[-\frac{((\overline{m} - \overline{n}) - \theta_{ij})^2}{2(\beta_j^2 + \sigma_i^2 + \tau_{ij}^2)}\right]
\end{align*}
Thus
\begin{align*}
	&\int\limits_{\mathbb{R}^2}a_{ij}(m_i, n_j)p(m_i, \overline{m_i})p(n_j, \overline{n_j})dm_idn_j = \frac{\alpha_{ij}\tau_{ij}}{\sqrt{\tau_{ij}^2 + \sigma_i^2 + \beta_j^2}}\exp{\left[-\frac{((\overline{m_i} - \overline{n_j}) - \theta_{ij})^2}{2(\tau_{ij}^2 + \sigma_i^2 + \beta_j^2)}\right]}
\end{align*}












\subsubsection*{\textbf{\textit{Derivation of (\ref{model_1_avg_pred_fitness})}}}
\begin{align*}
	\overline{W_i}(N_1, \dots, &N_u, M_i, \overline{n_1}, \dots, \overline{n_v}, \overline{m_i}) = \int\limits_{\mathbb{R}^{v+1}}^{}W_ip(m_i, \overline{m_i})\prod\limits_{j = 1}^{v}p(n_j, \overline{n_j})dm_i\prod\limits_{j = 1}^{v}dn_j \\
	&= \int\limits_{\mathbb{R}^{v+1}}^{}\left[\sum\limits_{j = 1}^{v}\left[e_{ij}a_{ij}(m_i, n_j)N_i\right] - d_i\right]p(m_i, \overline{m_i})\prod\limits_{j = 1}^{v}p(n_j, \overline{n_j})dm_i\prod\limits_{j = 1}^{v}dn_j \\
	&= \sum_{j=1}^{v}e_{ij}N_i\int\limits_{\mathbb{R}^{v+1}}^{}a_{ij}(m_i, n_j)p(m_i, \overline{m_i})\prod\limits_{j = 1}^{v}p(n_j, \overline{n_j})dm_i\prod\limits_{j = 1}^{v}dn_j \\
	&\ \ \ \ \ \ \ \ \ \ \ \ \ \ \ \ \ \ \ \ \ \ \ \ - d_i\int\limits_{\mathbb{R}^{v+1}}p(m_i, \overline{m_i})\prod\limits_{j = 1}^{v}p(n_j, \overline{n_j})dm_i\prod\limits_{j = 1}^{v}dn_j
\end{align*}
by the linearality of integrals.  Since $p(m_i, \overline{m_i})$ is independent of $n_k$ for $k = 1, \dots, v$ and $a_{ij}(m_i, n_j)$ and $p(n_j, \overline{n_j})$ are independent of $n_k$ for $k = 1, \dots, n_{j-1}, n_{j+1}, \dots, n_v$, then
\begin{align*}
	\int\limits_{\mathbb{R}^{v+1}}^{}&a_{ij}(m_i, n_j)p(m_i, \overline{m_i})\prod\limits_{j = 1}^{v}p(n_j, \overline{n_j})dm_i\prod\limits_{j = 1}^{v}dn_j \\
	&= \int\limits_{\mathbb{R}^2}a_{ij}(m_i, n_j)p(m_i, \overline{m_i})p(n_j, \overline{n_j})\left[\int\limits_{\mathbb{R}^{v-1}}\prod\limits_{\substack{k=1\\k\neq j}}^{v}p(n_k, \overline{n_k})\prod\limits_{\substack{k=1\\k\neq j}}^{v}dn_k\right]dm_idn_j
\end{align*}
However, all traits are assumed to have normal distributions, so
\begin{align*}
	\int\limits_{\mathbb{R}^{v-1}}\prod\limits_{\substack{k=1\\k\neq j}}^{v}p(n_k, \overline{n_k})\prod\limits_{\substack{k=1\\k\neq j}}^{v}dn_k = 1
\end{align*}
Thus
\begin{align*}
	\int\limits_{\mathbb{R}^{v+1}}^{}a_{ij}(m_i, n_j)p(m_i, \overline{m_i})\prod\limits_{j = 1}^{v}p(n_j, \overline{n_j})dm_i\prod\limits_{j = 1}^{v}dn_j &= \int\limits_{\mathbb{R}^2}a_{ij}(m_i, n_j)p(m_i, \overline{m_i})p(n_j, \overline{n_j})dm_idn_j \\
	&= \overline{a_{ij}}(\overline{m_i}, \overline{n_j})
\end{align*}
and
\begin{align*}
	d_i\left(\int\limits_{\mathbb{R}^{v+1}}p(m_i, \overline{m_i})\prod\limits_{j = 1}^{v}p(n_j, \overline{n_j})dm_i\prod\limits_{j = 1}^{v}dn_j\right) = d_i
\end{align*}
Thus,
\begin{align*}
	\overline{W_i}(N_1, \dots, &N_u, M_i, \overline{n_1}, \dots, \overline{n_v}, \overline{m_i}) = \sum\limits_{j = 1}^{v}\left[e_{ij}\overline{a_{ij}}(\overline{m_i}, \overline{n_j})N_i\right] - d_i
\end{align*}
















\subsubsection*{\textbf{\textit{Derivation of (\ref{model_1_avg_prey_fitness})}}}
\begin{align*}
	\overline{Y_j}(N_j, M_1, \dots, &M_v, \overline{n_j}, \overline{m_1}, \dots, \overline{m_v}) = \int\limits_{\mathbb{R}^{u+1}}^{}Y_jp(n_j, \overline{n_j})\prod\limits_{i = 1}^{u}p(m_i, \overline{m_i})dn_j\prod\limits_{i = 1}^{u}dm_i \\
	&= \int\limits_{\mathbb{R}^{u+1}}\left[r_j\left(1 - \frac{N_j}{K_j}\right) - \sum\limits_{i = 1}^{u}\left[a_{ij}(m_i, n_j)M_i\right]\right]p(n_j, \overline{n_j})\prod\limits_{i = 1}^{u}p(m_i, \overline{m_i})dn_j\prod\limits_{i = 1}^{u}dm_i \\
	&= r_j\left(1 - \frac{N_j}{K_j}\right)\int\limits_{\mathbb{R}^{u+1}}p(n_j, \overline{n_j})\prod\limits_{i = 1}^{u}p(m_i, \overline{m_i})dn_j\prod\limits_{i = 1}^{u}dm_i \\
	&\ \ \ \ \ \ \ \ \ \ \ \ \ \ \ - \sum\limits_{i=1}^{u}M_i\int\limits_{\mathbb{R}^{u+1}}a_{ij}(m_i, n_j)p(n_j, \overline{n_j})\prod\limits_{i = 1}^{u}p(m_i, \overline{m_i})dn_j\prod\limits_{i = 1}^{u}dm_i
\end{align*}
Similary to the derivation of (\ref{model_1_avg_pred_fitness}),
\begin{align*}
	r_j\left(1 - \frac{N_j}{K_j}\right)\left(\ \int\limits_{\mathbb{R}^{u+1}}p(n_j, \overline{n_j})\prod\limits_{i = 1}^{u}p(m_i, \overline{m_i})dn_j\prod\limits_{i = 1}^{u}dm_i\right) = r_j\left(1 - \frac{N_j}{K_j}\right)
\end{align*}
and
\begin{align*}
	\int\limits_{\mathbb{R}^{u+1}}a_{ij}(m_i, n_j)p(n_j, \overline{n_j})\prod\limits_{i = 1}^{u}p(m_i, \overline{m_i})dn_j\prod\limits_{i = 1}^{u}dm_i = \overline{a_{ij}}(\overline{m_i}, \overline{n_j})
\end{align*}
Thus,
\begin{align*}
	\overline{Y_j}(N_j, M_1, \dots, &M_v, \overline{n_j}, \overline{m_1}, \dots, \overline{m_v}) = r_j\left(1 - \frac{N_j}{K_j}\right) - \sum\limits_{i = 1}^{u}\left[\overline{a_{ij}}(\overline{m_i}, \overline{n_j})M_i\right]
\end{align*}















\subsubsection*{\textbf{\textit{Derivation of (\ref{average_growth_rate})}}}
\begin{align*}
	\overline{r_j}(\overline{n_j}) &= \int\limits_{\mathbb{R}}^{}r_j(n_j)p(n_j, \overline{n_j})dn_j \\
	&= \int\limits_{\mathbb{R}}\left(\rho_j\exp\left[-\frac{(n_j - \phi_j)^2}{2\gamma_j^2}\right]\right)\left(\frac{1}{\sqrt{2\pi\beta_j^2}}\exp\left[-\frac{(n_j - \overline{n_j})^2}{2\beta_j^2}\right]\right)dn_j \\
	&= \frac{\rho_j}{\beta_j\sqrt{2\pi}}\int\limits_{\mathbb{R}}\exp\left[-(an_j^2 + bn_j + c)\right]dn_j
\end{align*}
where $a = \dfrac{\beta_j^2 + \gamma_j^2}{2\beta_j^2\gamma_j^2}$, $b = -\left(\dfrac{\phi_j\beta_j^2 + \overline{n_j}\gamma_j^2}{\beta_j^2\gamma_j^2}\right)$, and $c = \dfrac{\phi_j^2\beta_j^2 + \overline{n_j}^2\gamma_j^2}{2\beta_j^2\gamma_j^2}$
\begin{align*}
	\implies \sqrt{\frac{\pi}{a}}\exp\left[\frac{b^2}{4a} - c\right] &= \frac{\beta_j\gamma_j\sqrt{2\pi}}{\sqrt{\beta_j^2 + \gamma_j^2}}\exp\left[-\frac{(\overline{n_j} - \phi_j)^2}{2(\beta_j^2 + \gamma_j^2)}\right] \\
	\implies \overline{r_j}(n_j) &= \frac{\rho_j\gamma_j}{\sqrt{B_j}}\exp\left[-\frac{(\overline{n_j} - \phi_j)^2}{2B_j}\right]
\end{align*}















\subsubsection*{\textbf{\textit{Derivation of (\ref{model_2_avg_prey_fitness})}}}
\begin{align*}
	\overline{Y_j}(N_j, M_1, \dots, &M_v, \overline{n_j}, \overline{m_1}, \dots, \overline{m_v}) = \int\limits_{\mathbb{R}^{u+1}}^{}Y_jp(n_j, \overline{n_j})\prod\limits_{i = 1}^{u}p(m_i, \overline{m_i})dn_j\prod\limits_{i = 1}^{u}dm_i \\
	&= \int\limits_{\mathbb{R}^{u+1}}\left[r_j(n_j)\left(1 - \frac{N_j}{K_j}\right) - \sum\limits_{i = 1}^{u}\left[a_{ij}(m_i, n_j)M_i\right]\right]p(n_j, \overline{n_j})\prod\limits_{i = 1}^{u}p(m_i, \overline{m_i})dn_j\prod\limits_{i = 1}^{u}dm_i \\
	&= \left(1 - \frac{N_j}{K_j}\right)\int\limits_{\mathbb{R}^{u+1}}r_j(n_j)p(n_j, \overline{n_j})\prod\limits_{i = 1}^{u}p(m_i, \overline{m_i})dn_j\prod\limits_{i = 1}^{u}dm_i \\
	&\ \ \ \ \ \ \ \ \ \ \ \ \ \ \ - \sum\limits_{i=1}^{u}M_i\int\limits_{\mathbb{R}^{u+1}}a_{ij}(m_i, n_j)p(n_j, \overline{n_j})\prod\limits_{i = 1}^{u}p(m_i, \overline{m_i})dn_j\prod\limits_{i = 1}^{u}dm_i
\end{align*}
Similary to the derivation of (\ref{model_1_avg_pred_fitness}), since $r_j(n_j)$ and $p(n_j, \overline{n_j})$ are not dependent on $n_k$ for $k = 1, \dots, \j-1, \j+1, \dots, v$, then 
\begin{align*}
	\int\limits_{\mathbb{R}^{u+1}}^{}r_j(n_j)p(m_i, \overline{m_i})&\prod\limits_{j = 1}^{v}p(n_j, \overline{n_j})dm_i\prod\limits_{j = 1}^{v}dn_j \\
	&= \int\limits_{\mathbb{R}}r_j(n_j)p(n_j, \overline{n_j})\left[\int\limits_{\mathbb{R}^{u}}p(m_i, \overline{m_i})\prod\limits_{\substack{k=1\\k\neq j}}^{v}p(n_k, \overline{n_k})dm_i\prod\limits_{\substack{k=1\\k\neq j}}^{v}dn_k\right]dn_j \\
	&= \int\limits_{\mathbb{R}}r_j(n_j)p(n_j, \overline{n_j})dn_j \\
	&= \overline{r_j}(\overline{n_j})
\end{align*}
and
\begin{align*}
	\int\limits_{\mathbb{R}^{u+1}}a_{ij}(m_i, n_j)p(n_j, \overline{n_j})\prod\limits_{i = 1}^{u}p(m_i, \overline{m_i})dn_j\prod\limits_{i = 1}^{u}dm_i = \overline{a_{ij}}(\overline{m_i}, \overline{n_j})
\end{align*}
Thus,
\begin{align*}
	\overline{Y_j}(N_j, M_1, \dots, &M_v, \overline{n_j}, \overline{m_1}, \dots, \overline{m_v}) = \overline{r_j}(\overline{n_j})\left(1 - \frac{N_j}{K_j}\right) - \sum\limits_{i = 1}^{u}\left[\overline{a_{ij}}(\overline{m_i}, \overline{n_j})M_i\right]
\end{align*}


































\subsection*{Appendix 2: Equilibria and Local Stability Analysis of (\ref{MODEL1})}
\begin{align*}
	f_1 = \frac{dM}{dt} &= M\left[e\overline{a}(\overline{m}, \overline{n})N - d\right] \\[5px]
	f_2 = \frac{dN}{dt} &= N\left[r\left(1 - \frac{N}{K}\right) - \overline{a}(\overline{m}, \overline{n})M\right] \\[5px]
	f_3 = \frac{d\overline{m}}{dt} &= \sigma_{G}^2\frac{eN(\theta - (\overline{m} - \overline{n}))}{A}\overline{a}(\overline{m}, \overline{n}) \\[5px]
	f_4 = \frac{d\overline{n}}{dt} &= \beta_{G}^2\frac{M(\theta - (\overline{m} - \overline{n}))}{A}\overline{a}(\overline{m}, \overline{n})
\end{align*}
\begin{align*}
	\begin{array}{ccccc}
		f_1 = 0 \implies& M = 0 \ \ \ &\text{or}&\ \ \ N = \dfrac{d}{e\overline{a}(\overline{m}, \overline{n})} \\[10px]
		f_2 = 0 \implies& N = 0 \ \ \ &\text{or}&\ \ \ M = \dfrac{r}{\overline{a}(\overline{m}, \overline{n})}\left(1 - \dfrac{N}{K}\right) \\[10px]
		f_3 = 0 \implies& N = 0 \ \ \ &\text{or}&\ \ \ \overline{m} - \overline{n} = \theta \\[10px]
		f_4 = 0 \implies& M = 0 \ \ \ &\text{or}&\ \ \ \overline{m} - \overline{n} = \theta
	\end{array}
\end{align*}
Clearly, $M = N = 0$ satisfies equilibrium, and $\overline{m}$ and $\overline{n}$ are arbitrary.  This gives the extinction equilibria.  Suppose $M = 0$ but $N \neq 0$.  Then $f_2 = 0 \implies N = K$ and $f_3 = 0 \implies \overline{m} - \overline{n} = \theta$.  This gives the exclusion equilibria.  Supposing $M \neq 0$ and $N \neq 0$, then $f_3 = f_4 = 0 \implies \overline{m} - \overline{n} = \theta \implies \overline{a}(\overline{m}, \overline{n}) = \frac{\alpha\tau}{\sqrt{A}} \implies N = \frac{d\sqrt{A}}{e\alpha\tau} \implies M = \frac{r\sqrt{A}}{\alpha\tau}\left(1 - \frac{\sqrt{A}}{Ke\alpha\tau}\right)$.  By exhaustion, this gives the only other set of equilibria: the coexistence equilibria.

To solve for local stability conditions, we find the eigenvalues of the community matrix evaluated at each equilbrium point.  First, note the partial derivatives of $f_1$, $f_2$, $f_3$, and $f_4$:
\begin{align*}
	\frac{\partial f_1}{\partial M} &= e\overline{a}(\overline{m}, \overline{n})N - d\\[5px]
	\frac{\partial f_1}{\partial N} &= Me\overline{a}(\overline{m}, \overline{n})\\[5px]
	\frac{\partial f_1}{\partial \overline{m}} &= \frac{MNe(\theta - (\overline{m} - \overline{n}))}{A}\overline{a}(\overline{m}, \overline{n}) \\[5px]
	\frac{\partial f_1}{\partial \overline{n}} &= \frac{MNe((\overline{m} - \overline{n}) - \theta)}{A}\overline{a}(\overline{m}, \overline{n})
\end{align*}
\begin{align*}
	\frac{\partial f_2}{\partial M} &= -N\overline{a}(\overline{m}, \overline{n}) \\[5px]
	\frac{\partial f_2}{\partial N} &= r\left(1 - \frac{2N}{K}\right) - M\overline{a}(\overline{m}, \overline{n})\\[5px]
	\frac{\partial f_2}{\partial \overline{m}} &= -\frac{MN(\theta - (\overline{m} - \overline{n}))}{A}\overline{a}(\overline{m}, \overline{n}) \\[5px]
	\frac{\partial f_2}{\partial \overline{n}} &= -\frac{MN((\overline{m} - \overline{n}) - \theta)}{A}\overline{a}(\overline{m}, \overline{n})
\end{align*}
\begin{align*}
	\frac{\partial f_3}{\partial M} &= 0 \\[5px]
	\frac{\partial f_3}{\partial N} &= \frac{\sigma_{G}^2e}{A}\overline{a}(\overline{m}, \overline{n})(\theta - (\overline{m} - \overline{n})) \\[5px]
	\frac{\partial f_3}{\partial \overline{m}} &= \frac{\sigma_G^2eN}{A}\overline{a}(\overline{m}, \overline{n})\left(\frac{(\theta - (\overline{m} - \overline{n}))^2}{A} - 1\right)\\[5px]
	\frac{\partial f_3}{\partial \overline{n}} &= \frac{\sigma_G^2eN}{A}\overline{a}(\overline{m}, \overline{n})\left(1 - \frac{(\theta - (\overline{m} - \overline{n}))^2}{A}\right)
\end{align*}
\begin{align*}
	\frac{\partial f_4}{\partial M} &= \frac{\beta_{G}^2}{A}\overline{a}(\overline{m}, \overline{n})(\theta - (\overline{m} - \overline{n})) \\[5px]
	\frac{\partial f_4}{\partial N} &= 0 \\[5px]
	\frac{\partial f_4}{\partial \overline{m}} &= \frac{\beta_G^2M}{A}\overline{a}(\overline{m}, \overline{n})\left(\frac{(\theta - (\overline{m} - \overline{n}))^2}{A} - 1\right)\\[5px]
	\frac{\partial f_4}{\partial \overline{n}} &= \frac{\beta_G^2M}{A}\overline{a}(\overline{m}, \overline{n})\left(1 - \frac{(\theta - (\overline{m} - \overline{n}))^2}{A}\right)
\end{align*}
Let the extinction, exclusion, and coexistence equilibrium be denoted as
\begin{align*}
	E_{\text{ext}} = (M^*, N^*, \overline{m}^*, \overline{n}^*) &= (0, 0, \mu^*, \nu^*) \\
	E_{\text{excl}} = (M^*, N^*, \overline{m}^*, \overline{n}^*) &= (0, K, \mu^* + \theta, \mu^*) \\
	E_{\text{coex}} = (M^*, N^*, \overline{m}^*, \overline{n}^*) &= \left(\frac{r\sqrt{A}}{\alpha\tau}\left(1 - \frac{N^*}{K}\right), \frac{d\sqrt{A}}{e\alpha\tau}, \mu^* + \theta, \mu^*\right)
\end{align*}
Then denote $J^*|_{E_{\text{ext}}}$, $J^*|_{E_{\text{excl}}}$, and $J^*|_{E_{\text{coex}}}$ as the community matrices evaluated at those points.
\begin{align*}
	J^*|_{E_{\text{ext}}} &= \left(\begin{array}{cccc}
		\dfrac{\partial f_1}{\partial M}\Big|_{E_{\text{ext}}} & \dfrac{\partial f_1}{\partial N}\Big|_{E_{\text{ext}}} & \dfrac{\partial f_1}{\partial \overline{m}}\Big|_{E_{\text{ext}}} & \dfrac{\partial f_1}{\partial \overline{n}}\Big|_{E_{\text{ext}}} \\[10px]
		\dfrac{\partial f_2}{\partial M}\Big|_{E_{\text{ext}}} & \dfrac{\partial f_2}{\partial N}\Big|_{E_{\text{ext}}} & \dfrac{\partial f_2}{\partial \overline{m}}\Big|_{E_{\text{ext}}} & \dfrac{\partial f_2}{\partial \overline{n}}\Big|_{E_{\text{ext}}} \\[10px]
		\dfrac{\partial f_3}{\partial M}\Big|_{E_{\text{ext}}} & \dfrac{\partial f_3}{\partial N}\Big|_{E_{\text{ext}}} & \dfrac{\partial f_3}{\partial \overline{m}}\Big|_{E_{\text{ext}}} & \dfrac{\partial f_3}{\partial \overline{n}}\Big|_{E_{\text{ext}}} \\[10px]
		\dfrac{\partial f_4}{\partial M}\Big|_{E_{\text{ext}}} & \dfrac{\partial f_4}{\partial N}\Big|_{E_{\text{ext}}} & \dfrac{\partial f_4}{\partial \overline{m}}\Big|_{E_{\text{ext}}} & \dfrac{\partial f_4}{\partial \overline{n}}\Big|_{E_{\text{ext}}}
	\end{array}\right) \\
	&= \left(\begin{array}{cccc}
		-d & 0 & 0 & 0 \\
		0 & r & 0 & 0 \\
		0 & \frac{\sigma_{G}^2e(\theta - (\mu^* - \nu^*))}{A}\overline{a}(\mu^*, \nu^*) & 0 & 0 \\
		\frac{\beta_{G}^2(\theta - (\mu^* - \nu^*))}{A}\overline{a}(\mu^*, \nu^*) & 0 & 0 & 0
	\end{array}\right)
\end{align*}
This is an upper-triangular matrix, and thus the eigenvalues are the entries on the main diagonal: $-d$, $r$, and $0$.  Since one of the eigenvalues, namely $r$, is positive, the $E_{\text{ext}}$ is locally unstable.
\begin{align*}
	J^*|_{E_{\text{excl}}} &= \left(\begin{array}{cccc}
		\dfrac{\partial f_1}{\partial M}\Big|_{E_{\text{excl}}} & \dfrac{\partial f_1}{\partial N}\Big|_{E_{\text{excl}}} & \dfrac{\partial f_1}{\partial \overline{m}}\Big|_{E_{\text{excl}}} & \dfrac{\partial f_1}{\partial \overline{n}}\Big|_{E_{\text{excl}}} \\[10px]
		\dfrac{\partial f_2}{\partial M}\Big|_{E_{\text{excl}}} & \dfrac{\partial f_2}{\partial N}\Big|_{E_{\text{excl}}} & \dfrac{\partial f_2}{\partial \overline{m}}\Big|_{E_{\text{excl}}} & \dfrac{\partial f_2}{\partial \overline{n}}\Big|_{E_{\text{excl}}} \\[10px]
		\dfrac{\partial f_3}{\partial M}\Big|_{E_{\text{excl}}} & \dfrac{\partial f_3}{\partial N}\Big|_{E_{\text{excl}}} & \dfrac{\partial f_3}{\partial \overline{m}}\Big|_{E_{\text{excl}}} & \dfrac{\partial f_3}{\partial \overline{n}}\Big|_{E_{\text{excl}}} \\[10px]
		\dfrac{\partial f_4}{\partial M}\Big|_{E_{\text{excl}}} & \dfrac{\partial f_4}{\partial N}\Big|_{E_{\text{excl}}} & \dfrac{\partial f_4}{\partial \overline{m}}\Big|_{E_{\text{excl}}} & \dfrac{\partial f_4}{\partial \overline{n}}\Big|_{E_{\text{excl}}}
	\end{array}\right) \\
	&= \left(\begin{array}{cccc}
		\frac{Ke\alpha\tau}{\sqrt{A}} - d & 0 & 0 & 0 \\
		-\frac{K\alpha\tau}{\sqrt{A}} & -r & 0 & 0 \\
		0 & 0 & -\frac{\sigma_G^2Ke\alpha\tau}{A^{3/2}} & \frac{\sigma_G^2Ke\alpha\tau}{A^{3/2}} \\
		0 & 0 & 0 & 0
	\end{array}\right)
\end{align*}
The eigenvalues are $\frac{Ke\alpha\tau}{\sqrt{A}} - d$, $-r$ and $-\frac{\sigma_G^2Ke\alpha\tau}{A^{3/2}}$, $0$.  (The eigenvalues are easily computable by swapping the third and fourth rows and columns to obtain an upper-triangular matrix).  All of these are non-positive if (\ref{exclusion_stability_MODEL1}) holds.
\begin{align*}
	J^*|_{E_{\text{coex}}} &= \left(\begin{array}{cccc}
		\dfrac{\partial f_1}{\partial M}\Big|_{E_{\text{coex}}} & \dfrac{\partial f_1}{\partial N}\Big|_{E_{\text{coex}}} & \dfrac{\partial f_1}{\partial \overline{m}}\Big|_{E_{\text{coex}}} & \dfrac{\partial f_1}{\partial \overline{n}}\Big|_{E_{\text{coex}}} \\[10px]
		\dfrac{\partial f_2}{\partial M}\Big|_{E_{\text{coex}}} & \dfrac{\partial f_2}{\partial N}\Big|_{E_{\text{coex}}} & \dfrac{\partial f_2}{\partial \overline{m}}\Big|_{E_{\text{coex}}} & \dfrac{\partial f_2}{\partial \overline{n}}\Big|_{E_{\text{coex}}} \\[10px]
		\dfrac{\partial f_3}{\partial M}\Big|_{E_{\text{coex}}} & \dfrac{\partial f_3}{\partial N}\Big|_{E_{\text{coex}}} & \dfrac{\partial f_3}{\partial \overline{m}}\Big|_{E_{\text{coex}}} & \dfrac{\partial f_3}{\partial \overline{n}}\Big|_{E_{\text{coex}}} \\[10px]
		\dfrac{\partial f_4}{\partial M}\Big|_{E_{\text{coex}}} & \dfrac{\partial f_4}{\partial N}\Big|_{E_{\text{coex}}} & \dfrac{\partial f_4}{\partial \overline{m}}\Big|_{E_{\text{coex}}} & \dfrac{\partial f_4}{\partial \overline{n}}\Big|_{E_{\text{coex}}}
	\end{array}\right) \\
	&= \left(\begin{array}{cccc}
		0 & er\left(1 - \frac{N^*}{K}\right) & 0 & 0 \\
		-\frac{d}{e} & -\frac{rN^*}{K} & 0 & 0 \\
		0 & 0 & -\frac{\sigma_G^2d}{A} & \frac{\sigma_G^2d}{A} \\
		0 & 0 & -\frac{\beta_G^2r}{A}\left(1 - \frac{N^*}{K}\right) & \frac{\beta_G^2r}{A}\left(1 - \frac{N^*}{K}\right)
	\end{array}\right)
\end{align*}
This is a block diagonal matrix, and so the eigenvalues can be calculated by finding the eigenvalues of each block.
\begin{align*}
	J_1 = \left(\begin{array}{cc}
		0 & er\left(1 - \frac{N^*}{K}\right)\\
		-\frac{d}{e} & -\frac{rN^*}{K}
	\end{array}\right)\ \ \ \ \text{and}\ \ \ \ J_2 = \left(\begin{array}{cc}
		-\frac{\sigma_G^2d}{A} & \frac{\sigma_G^2d}{A} \\
        -\frac{\beta_G^2r}{A}\left(1 - \frac{N^*}{K}\right) & \frac{\beta_G^2r}{A}\left(1 - \frac{N^*}{K}\right)
	\end{array}\right)
\end{align*}
The eigenvalues of $J_1$ are
\begin{align*}
	\lambda_{1,2} = \frac{1}{2}\left[-\frac{rN^*}{K} \pm \sqrt{\left(\frac{rN^*}{K}\right)^2 - 4rd\left(1 - \frac{N^*}{K}\right)}\ \right]
\end{align*}
Since $N^* < K$, $\sqrt{\left(\frac{rN^*}{K}\right)^2 - 4rd\left(1 - \frac{N^*}{K}\right)} < \left|\frac{rN^*}{K}\right|$, and thus $\text{Re}(\lambda_{1,2}) < 0$. \\

\noindent The eigenvalues of $J_2$ are
\begin{align*}
	\lambda_{3,4} = \frac{1}{2}\left[-\left(\frac{d\sigma_G^2 - r\beta_G^2\left(1 - \dfrac{N^*}{K}\right)}{A}\right) \pm \sqrt{\Delta}\right]
\end{align*}
where
\begin{align*}
	\Delta = \left(\frac{d\sigma_G^2 - r\beta_G^2\left(1 - \dfrac{N^*}{K}\right)}{A}\right)^2 - \left(\frac{4rd\sigma_G^2\beta_G^2\left(1 - \dfrac{N^*}{K}\right)}{A^2}\right)
\end{align*}
Again, since $N^* < K$, $\sqrt{\Delta} < \left|\dfrac{d\sigma_G^2 - r\beta_G^2\left(1 - \frac{N^*}{K}\right)}{A}\right|$, and thus $\text{Re}(\lambda_{3,4}) < 0$ if and only if (\ref{coexistence_stability_MODEL1}) holds.




































\subsection*{Appendix 3: Equilibria and Local Stability Analysis of (\ref{MODEL2})}
\begin{align*}
	f_1 = \frac{dM}{dt} &= M\left[e\overline{a}(\overline{m}, \overline{n})N - d\right] \\[5px]
	f_2 = \frac{dN}{dt} &= N\left[\overline{r}(\overline{n})\left(1 - \frac{N}{K}\right) - \overline{a}(\overline{m}, \overline{n})M\right] \\[5px]
	f_3 = \frac{d\overline{m}}{dt} &= \sigma_{G}^2\frac{eN(\theta - (\overline{m} - \overline{n}))}{A}\overline{a}(\overline{m}, \overline{n}) \\[5px]
	f_4 = \frac{d\overline{n}}{dt} &= \beta_{G}^2\Bigg[\overline{r}(\overline{n})\left(1 - \frac{N}{K}\right)\frac{\phi - \overline{n}}{B} + \frac{M(\theta - (\overline{m} - \overline{n}))}{A}\overline{a}(\overline{m}, \overline{n})\Bigg]
\end{align*}
\begin{align*}
	\begin{array}{cccrl}
		f_1 = 0 \implies& M = 0 \ \ \ &\text{or}&\ \ \ N &= \dfrac{d}{e\overline{a}(\overline{m}, \overline{n})} \\[10px]
		f_2 = 0 \implies& N = 0 \ \ \ &\text{or}&\ \ \ M &= \dfrac{\overline{r}(\overline{n})}{\overline{a}(\overline{m}, \overline{n})}\left(1 - \dfrac{N}{K}\right) \\[10px]
		f_3 = 0 \implies& N = 0 \ \ \ &\text{or}&\ \ \ \overline{m} - \overline{n} &= \theta \\[10px]
		f_4 = 0 \implies& M = 0 \ \ \ &\text{or}&\ \ \ \overline{r}(\overline{n})\left(1 - \dfrac{N}{K}\right)\dfrac{\phi - \overline{n}}{B} &= \dfrac{M((\overline{m} - \overline{n}) - \theta)}{A}\overline{a}(\overline{m}, \overline{n})
	\end{array}
\end{align*}
Clearly, $M = N = 0$ satisfies equilibrium, and $\overline{m}$ and $\overline{n}$ are arbitrary.  This gives the extinction equilibria.  Suppose $M = 0$ but $N \neq 0$.  Then $f_2 = 0 \implies N = K$ and $f_3 = 0 \implies \overline{m} - \overline{n} = \theta$.  This gives the exclusion equilibria.  Supposing $M \neq 0$ and $N \neq 0$, then $f_3 = 0 \implies \overline{m} - \overline{n} = \theta \implies \overline{a}(\overline{m}, \overline{n}) = \frac{\alpha\tau}{\sqrt{A}} \implies N = \frac{d\sqrt{A}}{e\alpha\tau}$.  Since $\overline{m} - \overline{n} = \theta$, then $f_4 = 0 \implies \overline{n} = \phi$ or $N = K$.  But if $N = K$, then $f_2 = 0 \implies M = 0$, a contradiction.  Thus $\overline{n} = \phi$, which implies $\overline{m} = \phi + \theta$ and $M = \frac{\rho\gamma\sqrt{A}}{\alpha\tau\sqrt{B}}\left(1 - \frac{\sqrt{A}}{Ke\alpha\tau}\right)$.  By exhaustion, this gives the only other equilibrium: the coexistence equilibrium point.

To solve for local stability conditions, we find the eigenvalues of the community matrix evaluated at each equilbrium point.  First, note the partial derivatives of $f_1$, $f_2$, $f_3$, and $f_4$:
\begin{align*}
	\frac{\partial f_1}{\partial M} &= e\overline{a}(\overline{m}, \overline{n})N - d\\[5px]
	\frac{\partial f_1}{\partial N} &= Me\overline{a}(\overline{m}, \overline{n})\\[5px]
	\frac{\partial f_1}{\partial \overline{m}} &= \frac{MNe(\theta - (\overline{m} - \overline{n}))}{A}\overline{a}(\overline{m}, \overline{n}) \\[5px]
	\frac{\partial f_1}{\partial \overline{n}} &= \frac{MNe((\overline{m} - \overline{n}) - \theta)}{A}\overline{a}(\overline{m}, \overline{n})
\end{align*}
\begin{align*}
	\frac{\partial f_2}{\partial M} &= -N\overline{a}(\overline{m}, \overline{n}) \\[5px]
	\frac{\partial f_2}{\partial N} &= \overline{r}(\overline{n})\left(1 - \frac{2N}{K}\right) - M\overline{a}(\overline{m}, \overline{n})\\[5px]
	\frac{\partial f_2}{\partial \overline{m}} &= -\frac{MN(\theta - (\overline{m} - \overline{n}))}{A}\overline{a}(\overline{m}, \overline{n}) \\[5px]
	\frac{\partial f_2}{\partial \overline{n}} &= N\left[\overline{r}(\overline{n})\left(1 - \frac{N}{K}\right)\frac{\phi - \overline{n}}{B}- \frac{M((\overline{m} - \overline{n}) - \theta)}{A}\overline{a}(\overline{m}, \overline{n})\right]
\end{align*}
\begin{align*}
	\frac{\partial f_3}{\partial M} &= 0 \\[5px]
	\frac{\partial f_3}{\partial N} &= \frac{\sigma_{G}^2e}{A}\overline{a}(\overline{m}, \overline{n})(\theta - (\overline{m} - \overline{n})) \\[5px]
	\frac{\partial f_3}{\partial \overline{m}} &= \frac{\sigma_G^2eN}{A}\overline{a}(\overline{m}, \overline{n})\left(\frac{(\theta - (\overline{m} - \overline{n}))^2}{A} - 1\right)\\[5px]
	\frac{\partial f_3}{\partial \overline{n}} &= \frac{\sigma_G^2eN}{A}\overline{a}(\overline{m}, \overline{n})\left(1 - \frac{(\theta - (\overline{m} - \overline{n}))^2}{A}\right)
\end{align*}
\begin{align*}
	\frac{\partial f_4}{\partial M} &= \frac{\beta_{G}^2}{A}\overline{a}(\overline{m}, \overline{n})(\theta - (\overline{m} - \overline{n})) \\[5px]
	\frac{\partial f_4}{\partial N} &= -\frac{\beta_G^2\overline{r}(\overline{n})}{K}\cdot\frac{\phi - \overline{n}}{B} \\[5px]
	\frac{\partial f_4}{\partial \overline{m}} &= \frac{\beta_G^2M}{A}\overline{a}(\overline{m}, \overline{n})\left(\frac{(\theta - (\overline{m} - \overline{n}))^2}{A} - 1\right)\\[5px]
	\frac{\partial f_4}{\partial \overline{n}} &= \beta_G^2\left[\frac{\overline{r}(\overline{n})}{B}\left(1 - \frac{N}{K}\right)\left(\frac{(\phi - \overline{n})^2}{B} - 1\right) + \frac{M}{A}\overline{a}(\overline{m}, \overline{n})\left(1 - \frac{(\theta - (\overline{m} - \overline{n}))^2}{A}\right)\right]
\end{align*}
Let the extinction, exclusion, and coexistence equilibria be denoted as
\begin{align*}
	E_{\text{ext}} = (M^*, N^*, \overline{m}^*, \overline{n}^*) &= (0, 0, \mu^*, \nu^*)\\
	E_{\text{excl}} = (M^*, N^*, \overline{m}^*, \overline{n}^*) &= (0, K, \mu^* + \theta, \mu^*) \\
	E_{\text{coex}} = (M^*, N^*, \overline{m}^*, \overline{n}^*) &= \left(\frac{\rho\gamma\sqrt{A}}{\alpha\tau\sqrt{B}}\left(1 - \frac{N^*}{K}\right), \frac{d\sqrt{A}}{e\alpha\tau}, \phi + \theta, \phi\right)
\end{align*}
Then denote $J^*|_{E_{\text{ext}}}$, $J^*|_{E_{\text{excl}}}$, and $J^*|_{E_{\text{coex}}}$ as the community matrices evaluated at those points.
\begin{align*}
	J^*|_{E_{\text{ext}}} &= \left(\begin{array}{cccc}
		\dfrac{\partial f_1}{\partial M}\Big|_{E_{\text{ext}}} & \dfrac{\partial f_1}{\partial N}\Big|_{E_{\text{ext}}} & \dfrac{\partial f_1}{\partial \overline{m}}\Big|_{E_{\text{ext}}} & \dfrac{\partial f_1}{\partial \overline{n}}\Big|_{E_{\text{ext}}} \\[10px]
		\dfrac{\partial f_2}{\partial M}\Big|_{E_{\text{ext}}} & \dfrac{\partial f_2}{\partial N}\Big|_{E_{\text{ext}}} & \dfrac{\partial f_2}{\partial \overline{m}}\Big|_{E_{\text{ext}}} & \dfrac{\partial f_2}{\partial \overline{n}}\Big|_{E_{\text{ext}}} \\[10px]
		\dfrac{\partial f_3}{\partial M}\Big|_{E_{\text{ext}}} & \dfrac{\partial f_3}{\partial N}\Big|_{E_{\text{ext}}} & \dfrac{\partial f_3}{\partial \overline{m}}\Big|_{E_{\text{ext}}} & \dfrac{\partial f_3}{\partial \overline{n}}\Big|_{E_{\text{ext}}} \\[10px]
		\dfrac{\partial f_4}{\partial M}\Big|_{E_{\text{ext}}} & \dfrac{\partial f_4}{\partial N}\Big|_{E_{\text{ext}}} & \dfrac{\partial f_4}{\partial \overline{m}}\Big|_{E_{\text{ext}}} & \dfrac{\partial f_4}{\partial \overline{n}}\Big|_{E_{\text{ext}}}
	\end{array}\right) \\
	&= \left(\begin{array}{cccc}
		-d & 0 & 0 & 0 \\[5px]
		0 & \frac{\rho\gamma}{\sqrt{B}} & 0 & 0 \\[5px]
		0 & \frac{\sigma_{G}^2e(\theta - (\mu^* - \nu^*))\overline{a}(\mu^*, \nu^*)}{A} & 0 & 0 \\[5px]
		\frac{\beta_{G}^2(\theta - (\mu^* - \nu^*))\overline{a}(\mu^*, \nu^*)}{A} & -\frac{\beta_G^2\overline{r}(\nu^*)}{K}\cdot\frac{\phi - \nu^*}{B} & 0 & \beta_G^2\frac{\overline{r}(\overline{n})}{B}\left(\frac{(\phi - \overline{n})^2}{B} - 1\right)
	\end{array}\right)
\end{align*}
This is an upper-triangular matrix, and thus the eigenvalues are the entries on the main diagonal: $-d$, $\frac{\rho\gamma}{\sqrt{B}}$, $0$, and $\beta_G^2\frac{\overline{r}(\overline{n})}{B}\left(\frac{(\phi - \overline{n})^2}{B} - 1\right)$.  Since one of the eigenvalues, namely $\frac{\rho\gamma}{\sqrt{B}}$, is positive, the $E_{\text{ext}}$ is locally unstable.
\begin{align*}
	J^*|_{E_{\text{excl}}} &= \left(\begin{array}{cccc}
		\dfrac{\partial f_1}{\partial M}\Big|_{E_{\text{excl}}} & \dfrac{\partial f_1}{\partial N}\Big|_{E_{\text{excl}}} & \dfrac{\partial f_1}{\partial \overline{m}}\Big|_{E_{\text{excl}}} & \dfrac{\partial f_1}{\partial \overline{n}}\Big|_{E_{\text{excl}}} \\[10px]
		\dfrac{\partial f_2}{\partial M}\Big|_{E_{\text{excl}}} & \dfrac{\partial f_2}{\partial N}\Big|_{E_{\text{excl}}} & \dfrac{\partial f_2}{\partial \overline{m}}\Big|_{E_{\text{excl}}} & \dfrac{\partial f_2}{\partial \overline{n}}\Big|_{E_{\text{excl}}} \\[10px]
		\dfrac{\partial f_3}{\partial M}\Big|_{E_{\text{excl}}} & \dfrac{\partial f_3}{\partial N}\Big|_{E_{\text{excl}}} & \dfrac{\partial f_3}{\partial \overline{m}}\Big|_{E_{\text{excl}}} & \dfrac{\partial f_3}{\partial \overline{n}}\Big|_{E_{\text{excl}}} \\[10px]
		\dfrac{\partial f_4}{\partial M}\Big|_{E_{\text{excl}}} & \dfrac{\partial f_4}{\partial N}\Big|_{E_{\text{excl}}} & \dfrac{\partial f_4}{\partial \overline{m}}\Big|_{E_{\text{excl}}} & \dfrac{\partial f_4}{\partial \overline{n}}\Big|_{E_{\text{excl}}}
	\end{array}\right) \\
	&= \left(\begin{array}{cccc}
		\frac{Ke\alpha\tau}{\sqrt{A}} - d & 0 & 0 & 0 \\
		-\frac{K\alpha\tau}{\sqrt{A}} & -\frac{\rho\gamma}{\sqrt{B}} & 0 & 0 \\
		0 & 0 & -\frac{\sigma_G^2Ke\alpha\tau}{A^{3/2}} & \frac{\sigma_G^2Ke\alpha\tau}{A^{3/2}} \\
		0 & -\frac{\beta_G^2\rho\gamma}{K\sqrt{B}}\cdot\frac{\phi - \mu^*}{B} & 0 & 0
	\end{array}\right)
\end{align*}
The eigenvalues are $\frac{Ke\alpha\tau}{\sqrt{A}} - d$, $-\frac{\rho\gamma}{\sqrt{B}}$ and $-\frac{\sigma_G^2Ke\alpha\tau}{A^{3/2}}$, $0$.  (The eigenvalues are easily computable by swapping the third and fourth rows and columns to obtain an upper-triangular matrix).  All of these are non-positive if (\ref{exclusion_stability_MODEL1}) holds.
\begin{align*}
	J^*|_{E_{\text{coex}}} &= \left(\begin{array}{cccc}
		\dfrac{\partial f_1}{\partial M}\Big|_{E_{\text{coex}}} & \dfrac{\partial f_1}{\partial N}\Big|_{E_{\text{coex}}} & \dfrac{\partial f_1}{\partial \overline{m}}\Big|_{E_{\text{coex}}} & \dfrac{\partial f_1}{\partial \overline{n}}\Big|_{E_{\text{coex}}} \\[10px]
		\dfrac{\partial f_2}{\partial M}\Big|_{E_{\text{coex}}} & \dfrac{\partial f_2}{\partial N}\Big|_{E_{\text{coex}}} & \dfrac{\partial f_2}{\partial \overline{m}}\Big|_{E_{\text{coex}}} & \dfrac{\partial f_2}{\partial \overline{n}}\Big|_{E_{\text{coex}}} \\[10px]
		\dfrac{\partial f_3}{\partial M}\Big|_{E_{\text{coex}}} & \dfrac{\partial f_3}{\partial N}\Big|_{E_{\text{coex}}} & \dfrac{\partial f_3}{\partial \overline{m}}\Big|_{E_{\text{coex}}} & \dfrac{\partial f_3}{\partial \overline{n}}\Big|_{E_{\text{coex}}} \\[10px]
		\dfrac{\partial f_4}{\partial M}\Big|_{E_{\text{coex}}} & \dfrac{\partial f_4}{\partial N}\Big|_{E_{\text{coex}}} & \dfrac{\partial f_4}{\partial \overline{m}}\Big|_{E_{\text{coex}}} & \dfrac{\partial f_4}{\partial \overline{n}}\Big|_{E_{\text{coex}}}
	\end{array}\right) \\
	&= \left(\begin{array}{cccc}
		0 & \frac{e\rho\gamma}{\sqrt{B}}\left(1 - \frac{N^*}{K}\right) & 0 & 0 \\
		-\frac{d}{e} & -\frac{\rho\gamma N^*}{K\sqrt{B}} & 0 & 0 \\
		0 & 0 & -\frac{\sigma_G^2d}{A} & \frac{\sigma_G^2d}{A} \\
		0 & 0 & -\frac{\beta_G^2\rho\gamma}{A\sqrt{B}}\left(1 - \frac{N^*}{K}\right) & \frac{\beta_G^2\rho\gamma}{\sqrt{B}}\left(1 - \frac{N^*}{K}\right)\left(\frac{1}{A} - \frac{1}{B}\right)
	\end{array}\right)
\end{align*}
This is a block diagonal matrix, and so the eigenvalues can be calculated by finding the eigenvalues of each block.
\begin{align*}
	J_1 = \left(\begin{array}{cc}
		0 & \frac{e\rho\gamma}{\sqrt{B}}\left(1 - \frac{N^*}{K}\right)\\
		-\frac{d}{e} & -\frac{\rho\gamma N^*}{K\sqrt{B}}
	\end{array}\right)\ \ \ \ \text{and}\ \ \ \ J_2 = \left(\begin{array}{cc}
		-\frac{\sigma_G^2d}{A} & \frac{\sigma_G^2d}{A} \\
        -\frac{\beta_G^2\rho\gamma}{A\sqrt{B}}\left(1 - \frac{N^*}{K}\right) & \frac{\beta_G^2\rho\gamma}{\sqrt{B}}\left(1 - \frac{N^*}{K}\right)\left(\frac{1}{A} - \frac{1}{B}\right)
	\end{array}\right)
\end{align*}
The eigenvalues of $J_1$ are
\begin{align*}
	\lambda_{1,2} = \frac{1}{2}\left[-\frac{\rho\gamma N^*}{K\sqrt{B}} \pm \sqrt{\left(\frac{\rho\gamma N^*}{K\sqrt{B}}\right)^2 - 4\rho\gamma d\left(1 - \frac{N^*}{K}\right)}\ \right]
\end{align*}
Since $N^* < K$, $\sqrt{\left(\frac{\rho\gamma N^*}{K\sqrt{B}}\right)^2 - 4\rho\gamma d\left(1 - \frac{N^*}{K}\right)} < \left|\frac{\rho\gamma N^*}{K\sqrt{B}}\right|$, and thus $\text{Re}(\lambda_{1,2}) < 0$. \\

\noindent For simplicity, let
\begin{align*}
	C &= \frac{d\sigma_G^2}{A} \\
	\text{and} \ \ \ D &= \frac{\beta_G^2\rho\gamma}{\sqrt{B}}\left(1 - \frac{N^*}{K}\right)
\end{align*}
then the eigenvalues of $J_2$ are
\begin{align*}
	\lambda_{3,4} = \frac{1}{2}\left[-\left(C + D\left[\frac{1}{B} - \frac{1}{A} \right]\right) \pm \sqrt{\Delta}\ \right]
\end{align*}
where
\begin{align*}
	\Delta = \left(C + D\left[\frac{1}{B} - \frac{1}{A} \right]\right)^2 - \left(\frac{4CD}{B}\right)
\end{align*}
Again, $N^* < K \implies D > 0$ implies that if $\sqrt{\Delta} \in \mathbb{R}$, then $\sqrt{\Delta} < \left|\left(C + D\left[\frac{1}{B} - \frac{1}{A} \right]\right)\right|$, and thus $\text{Re}(\lambda_3) < 0$ if and only if $\text{Re}(\lambda_4) < 0$ if and only if (\ref{coexistence_stability_MODEL2}) holds.

Note that a Hopf Bifurcation occurs when parameters are shifted in such a way that neither (\ref{coexistence_stability_MODEL2}) nor (\ref{exclusion_stability_MODEL1}) holds.  As certain parameters are shifted, both $\lambda_3$ and $\lambda_4$ may cross the complex axis from the negative real component of the complex plane to positive real component.







\FloatBarrier
\pagebreak
\bibliographystyle{amsplain}
\begin{thebibliography}{10}

\bibitem{Schreiber_2011}
Schreiber, S.~J., B\"urger,  R., and Bolnick,  D.~I.
The Community Effects of Phenotypic and Genetic Variation within a Predator Population.
\emph{Ecology}
2011,  92(8):526-543. 

\bibitem{Lande_1976}
Lande, R.
Natural Selection and Random Genetic Drift in Phentypic Evolution.
\emph{Society for the Study of Evolution}
1976, 30(2):314-334.

\bibitem{Abrams_Matsuda_1997}
Abrams, P.~A. and Matsuda, H.
Prey Adaptation as a Cause of Predator-Prey Cycles.
\emph{Society for the Study of Evolution}
1997, 51:1742-1750.

\bibitem{Khibnik_Kondrashov_1997}
Khibnik, A.~I. and Kondrashov, A.~S.
Three Mechanisms of Red Queen Dynamics.
\emph{Procedings of the Royal Society of London}
1997, 264:1049-1056.

\bibitem{Saloniemi_1993}
Saloniemi, I.
A Coevolutionary Predator-Prey Model with Quantitative Characters.
\emph{The American Naturalist}
1993, 141:880-896.

\end{thebibliography}



\vfill


% \begin{itemize}
% 	% \item Dr. Jing Li, our mentor and advisor
% 	% \item Dr. Casey terHorst, out Biology consultant
% 	% \item Dr. Helena Noronha, PUMP organizer
% 	% \item The National Science Foundation
% 	% \item The National Math Alliance
% 	% \item The Pacific Math Alliance
% 	% \item California State University PUMP (Preparing Undergraduates through Mentoring toward PhDs)
% 	% \item[$\star$] Dr. Jing Li, our mentor and advisor, for her unwavering support, constructive criticism, harsh truths, gentle guidance, and continuous excitement
% 	% \item[$\star$] Dr. Casey terHorst, our Biology consultant, for his intuition and insight, and for introducing us to the concept of Eco-Evo Feedback
% 	% \item[$\star$] Our friends and family, for their constant and consistent love
% \end{itemize}

\end{document}

%------------------------------------------------------------------------------
% End of journal.tex
%------------------------------------------------------------------------------
